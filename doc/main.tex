%% LyX 1.3 created this file.  For more info, see http://www.lyx.org/.
%% Do not edit unless you really know what you are doing.
\documentclass[11pt,twoside,brazil]{pep}
\usepackage{a4wide}
\usepackage{geometry}
\geometry{verbose,a4paper}
\pagestyle{headings}
\usepackage{float}
\usepackage{multirow}
\usepackage[T1]{fontenc}
\usepackage[utf8x]{inputenc} 
%\usepackage[utf8]{inputenc}
\usepackage{graphicx}
%\usepackage[breaklinks=true]{hyperref}
%\usepackage{cite}
\usepackage{breakcites}
% \usepackage[latin1]{inputenc}
\usepackage{array}
\newcolumntype{P}[1]{>{\centering\arraybackslash}p{#1}}
%\makesavenoteenv{tabular}
%\makesavenoteenv{table}
\usepackage{tablefootnote}
\usepackage{scrextend}
\usepackage{hyperref}
\usepackage{multirow}
\usepackage[toc,page]{appendix}
\usepackage{enumerate}

\graphicspath{{images/}}

\makeatletter

%%%%%%%%%%%%%%%%%%%%%%%%%%%%%% LyX specific LaTeX commands.
%% Bold symbol macro for standard LaTeX users
\providecommand{\boldsymbol}[1]{\mbox{\boldmath $#1$}}

\floatstyle{ruled}
%\newfloat{algorithm}{tbp}{loa}
%\floatname{algorithm}{Algorithm}

%%%%%%%%%%%%%%%%%%%%%%%%%%%%%% Textclass specific LaTeX commands.
\usepackage{verbatim}

\usepackage{babel}
%\DeclareUnicodeCharacter{00A0}{-}
%\DeclareUnicodeCharacter{00A0}{~}
\makeatother

\setcounter{secnumdepth}{3}
\setcounter{tocdepth}{3}


\begin{document}

\title{Reconhecimeno de Placas de Carro em Tempo Real}

 
\author{Daniel Antoniazzi Amarante e Matthias Oliveira Nunes}

\Advisor{Prof. Roland Teodorowitsch}

\maketitle

\maketitlerosto

%\begin{resumo}
%Mussum Ipsum, cacilds vidis litro abertis. Viva Forevis aptent taciti 
sociosqu ad litora torquent Suco de cevadiss deixa as pessoas mais 
interessantiss. Todo mundo vê os porris que eu tomo, mas ninguém vê os 
tombis que eu levo! Delegadis gente finis, bibendum egestas augue arcu 
ut est.

Praesent malesuada urna nisi, quis volutpat erat hendrerit non. Nam 
vulputate dapibus. Quem num gosta di mé, boa gente num é. Sapien in 
monti palavris qui num significa nadis i pareci latim. Quem num gosti 
di mum que vai caçá sua turmis!

Mauris nec dolor in eros commodo tempor. Aenean aliquam molestie leo, 
vitae iaculis nisl. Detraxit consequat et quo num tendi nada. Posuere 
libero varius.  Nullam a nisl ut ante blandit hendrerit. Aenean sit 
amet nisi. Ta deprimidis, eu conheço uma cachacis que pode alegrar sua 
vidis.” 

%\end{resumo}

%\begin{abstract}
%\input{abstract}
%\end{abstract}

\tableofcontents{}

\listoffigures

\listoftables

\chapter{Introdução}
\label{cha:intro}
O controle e identificação de veículos é usado nas mais diversas áreas, indo
desde serviços de pagamentos automatizados, como pedágios, até aplicações mais
críticas, como segurança de fronteiras, sistemas de vigilância de
tráfego~\cite{ahmad2015automatic} e sistema de busca por carros roubados.
Uma solução para identificar é, inclusive,  parte do plano de governo do atual prefeito eleito de Porto Alegre
Nelson Marchezan. Ele pretende utilizar os sistemas de controle de velocidade da cidade para também
monitorar as placas de carros com o objetivo de identificar carros roubados~\cite{psdb2016marchezan}.
Com o crescimento constante da frota de carros no Brasil, aplicações para
auxiliar neste trabalho tornam-se cada vez
mais nescessárias. Com isso em mente, propõe-se neste trabalho uma solução de
aplicação embarcada de reconhecimento de placas, visando a
crescente nescessidade de controle na área e as peculiaridades das placas
automotivas brasileiras, que nos impossibilitam de utilizar ferramentas
configuradas para placas estrangeiras, nescessitando pesquisas locais neste
tema.

O reconhecimento automático de placas de carros (\emph{Automatic License-Plate
Recognition}, ALPR) é a extração das informações das placas de veículos a partir
de uma imagem ou de uma sequência de imagens. A sua utilização na vida real
precisa de um processamento rápido e bem sucedido de placas sob diferentes
condições ambientais. Deve-se considerar as diferenças entre as placas de
diferentes nações, que terão cores, fontes, símbolos, padrões e línguas
diferentes. Também é preciso superar casos onde as placas possam estar
parcialmente cobertas com sujeira, luzes e acessórios dos
carros, e também a iluminação do ambiente e qualidade
da imagem adquirida.~\cite{s2013automatic}

A solução proposta neste trabalho tem como diferencial o fato de permitir uma
análise e processamento das imagens em tempo real. Isso será realizado
utilizando um software embarcado, que estará coletando as imagens ao mesmo tempo
que as analisa e as envia para um servidor. Com essa abordagem, será possível
que esses dados sejam úteis para uma tomada de decisão imediata do usuário.

Um exemplo de uma aplicação, onde a velocidade e a disponibilidade imediata das informações é
crucial para a viabilidade do produto, seria em um \emph{software} de identificação
de carros roubados. O sistema analisaria as placas dos carros que trafegam em uma
rodovia e identificaria quais daqueles carros tem placas que correspondem a placas
de carros roubados. Para que o \emph{software} seja eficiente é nescessário
que o processamento seja feito em tempo real, qualquer atraso na identificação do
veículo permite que ele se distancie muito, impedindo que as autoridades reajam a tempo.

Uma abordagem alternativa seria fazer a gravação das imagens em um momento, e o
processamento em outro. Entretanto, desta maneira, a utilidade do
\emph{software} seria limitada, pois na maioria das aplicações reais estes dados
precisam estar disponíveis imediatamente.

Com o resultado da implementação proposta será feita uma prova de conceito
utilizando o estacionamento da PUCRS, se for possível adquirir permissão.
Posicionando o computador com a câmera em um ponto estratégico do
estacionamento, serão analisados os carros que passarem. O programa então
coletará informações referentes ao número de carros que passaram por aquele ponto
e suas respectivas placas.

Este trabalho está organizado da seguinte maneira: No capítulo~\ref{cha:trab},
serão apresentados trabalhos relacionados aos temas de reconhecimento de placas
e contagem de carros. No capítulo~\ref{cha:modelo}, será apresentado o modelo
proposto. No capítulo~\ref{cha:implementacao}, será especificado os passos a serem
seguidos no desenvolvimento da aplicação, classificando os algoritmos a serem utilizados
em cada etapa. No capítulo~\ref{cha:configuracao}, será demonstrada como deve ser feita
a configuração e instalação das ferramentas nescessárias para o desenvolvimento do trabalho.
 E nos capítulos~\ref{cha:crono}  e~\ref{cha:recursos}, serão mostrados
o cronograma planejado e os recursos nescessários para a conclusão deste
trabalho respectivamente.


\chapter{Trabalhos Relacionados}
\label{cha:trab}
Uma solução para detectar e reconhecer placas de licenciamento brasileiras foi
proposta por Serro~\cite{serro2012deteccao} na PUCRS\@. Neste trabalho foram
utilizadas técnicas de segmentação de imagens, histograma, cisalhamento de
imagens e reconhecimento ótico de caracteres. A metodologia utilizada consistiu das
seguintes etapas, calibração do sistema
para definir a região de interesse e o ângulo de cisalhamento, detecção da
placa, segmentação dos caracteres e aplicação do reconhecedor ótico de
caracteres.

Com a solução proposta por Serro~\cite{serro2012deteccao}  foi obtida uma taxa de
acerto de aproximadamente 54\%, com tempo médio de execução de 0,062 segundos por imagem.
A baixa taxa de acerto pode ter sido por problemas de foco e nitidez, o tamanho da placa em
\emph{pixels} nas imagens e a existência de outros objetos em cena. Todos esses
problemas foram citados no desenvolvimento do projeto.

Em Ahmad et al.~\cite{ahmad2015automatic} foi feito um estudo comparativo dos
sistemas de reconhecimento de placas automotivas automáticos. Segundo estes
autores, o processo de ler o conteúdo de uma placa passa por três estágios. O
primeiro é a localização ou extração da placa, que consiste no processo de
localizar a placa do carro na imagem. O segundo estágio é a separação dos
caracteres, onde cada caractere individual é separado dos outros para
reconhecimento. E o terceiro e último estágio é o reconhecimento do caractere em
si, onde os caracteres extraídos da imagem são identificados.

Neste trabalho foram implementados três diferentes métodos de localização de
placa e dois diferentes métodos de reconhecimento de caracteres, resultando em 6
diferentes abordagens para o reconhecimento de placas. Todas essas combinações
foram então testadas contra diferentes conjuntos de dados.

Os resultados obtidos por Ahmad et al.~\cite{ahmad2015automatic} na experimentação não
foram muito animadores, variando entre 20 e 40 por cento.
Um dos motivos para os maus resultados foi a variedade de parâmetros nas imagens do conjunto
de dados de teste. Tais parâmetros incluem
variações na distância, ângulo, iluminação e ambiente. Estes erros poderiam ser
mitigados em sistemas reais com uma câmera com resolução fixa e de boa
qualidade. A variação do tamanho da placa afetou o desempenho de alguns
algoritmos, mas em uma câmera fixa é possível obter uma consistência e conseguir
resultados mais aceitáveis.

Outros motivos para a baixa taxa de acerto foram a falta de pré-processamento
das imagens, que a análise não considerou, e a utilização de mais dados de
aprendizado para o reconhecimento ótico de caracteres.

Em Abtahi et al.~\cite{abtahi2015deep} foram feitas novas abordagens para a
segmentação de caracteres em imagens. De acordo com eles, o método padrão de
segmentação baseado em projeção sofre com variações consideráveis na região da
placa ao redor dos caracteres, portanto estes autores propuseram duas abordagens.
A primeira é feita adaptando um método de aprendizado por reforço, criando um agente que
consiga achar os melhores caminhos para a segmentação.  A segunda abordagem usa
um método híbrido que utiliza a simplicidade e velocidade do método de projeção,
mas com o poder do aprendizado por reforço.

De acordo com Wafy e Madbouly~\cite{wafy2016efficient}, o reconhecimento de uma
placa consiste em dois mecanismos principais: detecção de uma placa e em seguida
a sua identificação. O algoritmo proposto nesse artigo, faz os dois passos e se
baseia na distribuição semi-simétrica dos pontos de canto nas imagens de carros
e placas, e nas características morfológicas da região da placa. Essa solução
teve uma taxa de acerto de 97,5\% no processo de detecção e 92,8\% na
identificação, com o maior tempo de execução para um dos processos sendo de
0,3s. Com estes resultados seria possível utilizar este método em aplicações de
tempo real.

Com relação ao uso de sistema embarcado para executar o reconhecimento, Arth et
al.~\cite{arth2007real} trabalharam no desenvolvimento de um sistema de reconhecimento
de placas de carro em um processador de digital de sinal (\emph{Digital Signal Processor}, DSP).
\emph{DSP} são microprocessadores especializados em processamento digital de sinal utilizados para
processar sinais como áudio ou vídeo em tempo real.~\cite{yovits1993advances} O processador utilizado
neste trabalho específico foi um \emph{Texas Instruments C64} com \emph{1MB} de \emph{cache RAM} e um outro bloco
de memória mais lento, \emph{SDRAM} de \emph{16MB}. O processador não possui camera integrada mas permite a conexão de
uma fonte de vídeo analógica ou digital. Na solução implementada foi utilizada uma camera com resolução de \emph{352x288 pixels}.

Com sua implementação, Arth et al.~\cite{arth2007real} foram capazes de conseguir localizar a placa em
\emph{7.30 ms}, levando mais aproximadamente \emph{1 ms} para identificar cada caractere. Não é informado no artigo
a taxa de sucesso de cada reconhecimento. Os autores ainda concluem que por o tempo de detecção da placa ser superior
ao tempo de reconhecimento dos caracteres, este algoritmo deve ser melhorado.

Analisando os trabalhos feitos na área, nota-se que por mais que os algoritmos variem,
a base da detecção de placas permanece parecida. Eles costumam ser divididos em pelo
menos duas partes, a localização da placa e a detecção dos caracteres. Inclusive~\cite{ahmad2015automatic}
misturou diferentes algoritmos, utilizando a localização de um e a detecção de
outro, demonstrando que os dois passos são bem independentes.

Pode-se notar que o reconhecimento de placas de carros é uma área bastante estudada,
com diversas abordagens diferentes e grande variação de resultados. Entretanto,
um problema que ainda existe, é a grande diferença nas placas de diferentes países,
no estilo, fonte, caracteres utilizados e padrão do texto.

O \emph{software open source Openalpr}\footnote{https://github.com/openalpr/openalpr}
criou uma solução de reconhecimento de placas que permite que a comunidade contribua
treinando o reconhecedor de caracteres a reconhecer as placas de seus países para contornar
esse problema.


\chapter{Modelo}
\label{cha:modelo}

O objetivo deste trabalho é desenvolver um software embarcado em um Raspberry Pi
equipado com módulo de câmera que seja capaz de reconhecer veículos, contá-los e
identificá-los baseado-se em sua placa. Todo o processamento deve acontecer em
tempo real com base nas imagens da câmera captadas no momento e a informação
coletada e processada deve ser enviada para um servidor que possa utilizar estes
dados.

O fluxo esperado do software é o seguinte. O computador Raspberry pi ficará postado
em uma rota de fluxo frequente de carros, posicionado de maneira que consiga capturar
imagens das placas em boa qualidade com sua camera. Será nescessário o módulo de câmera
e de bateria, ou haver uma fonte de energia por perto. Ao capturar as imagens, o computador
irá processá-las localmente, seguindo os passos do reconhecimento de placas e contagem de carros.
Após obtida, a informação será enviada para um servidor simples que irá armazená-la.

Segundo~\cite{ahmad2015automatic}, o reconhecimento de placas automotivas requer
três passos, a localização da placa, a separação dos caracteres e o
reconhecimento dos caracteres. Para a localização da placa e separação dos
caracteres será utilizada a biblioteca OpenCV e a linguagem de programação C,
C++ ou Python, pois são as linguagens mais usadas para OpenCV, a melhor
abordagem ainda será analisada. Para o reconhecimento dos caracteres será
estudada a possibilidade de utilizar o software Tesseract ou criar uma solução
própria. As escolhas a serem feitas tem como objetivo maximizar os resultados ao
final do trabalho, tentando criar um balanço entre facilidade de implementação e
qualidade do reconhecimento.

O trabalho será realizado em etapas, tendo como objetivo intermediário conseguir
fazer o reconhecimento e a contagem dos carros, e como objetivo final a
identificação das placas.

\section{Raspberry Pi}
\label{sec:raspi}

Raspberry Pi é um computador construído em uma placa de circuito do tamanho de
um cartão de crédito desenvolvido pela Raspberry Pi
Foundation\footnote{https://www.raspberrypi.org/}.

\section{OpenCV}
\label{sec:opencv}

OpenCV (Open Source Computer Vision Library) é uma biblioteca open source de
visão computacional e aprendizado de máquina. Contém mais de 2500 algoritmos
otimizados nessas áreas, incluindo algoritmos clássicos e recentes. A biblioteca
é escrita nativamente em C++, e dispõe de interfaces para C, C++, Python, Java e
MATLAB, suportando os sistemas operacionais Windows, Linux, Android e Mac
OS.\footnote{http://opencv.org/}

\section{OCR}
\label{sec:ocr}

Reconhecimento Ótico de Caracteres(OCR) consiste da conversão de textos em
formato de imagem para o formato reconhecido por maquina. É o método mais
eficiente para fazer o processamento de imagem para
texto.~\cite{mohit2015designing}

Uma ferramenta conhecida de OCR é o
Tesseract\footnote{https://github.com/tesseract-ocr/tesseract}. É uma ferramenta
open source de reconhecimento ótico de caracteres que suporta múltiplas linguas.
É essencialmente um algoritmo de comparação de templates, e as amostras de
caracteres podem ser auto-treinados.~\cite{ho2016intelligent}



\chapter{Cronograma}
\label{cha:crono}
O desenvolvimento deste trabalho prevê as seguintes atividades.
Estas atividades tem como objetivo definir o modelo que será implementado
e iniciar a implementação

\begin{table}[H]
\centering
\label{tab:crono}
\begin{tabular}{c|c|c|c}
    & Setembro & Outubro & Novembro \\ \hline
I   & X        &         &          \\
II  &          & X       &          \\
III &          & X       &          \\
IV  &          & X       &          \\
V   &          &         & X       
\end{tabular}
\caption{Cronograma proposto}
\end{table}

\begin{enumerate}[I]
	\item Pesquisar e estudar trabalhos relacionados atuais.
	\item Estudar a biblioteca OpenCV\@.
	\item Estudar a ferramenta para reconhecimento ótico de caracteres Tesseract.
	\item Estudar o Raspberry Pi e seu módulo de câmera.
	\item Definir modelo e iniciar implementação.
	\item Integrar o OpenCV e o Raspberry Pi e implementar uma prova de conceito que utilize os dois.
	\item Implementar uma aplicação que localize a placa de um carro em uma imagem utilizando OpenCV.
	\item Escrever o volume final de TCI.
\end{enumerate}



\chapter{Recursos Necessários}
\label{cha:recursos}
Para o desenvolvimento do trabalho serão nescessários os seguintes recursos. 
Estão aqui discriminadas as peças físicas de \emph{hardware} que integrarão o produto final desenvolvido,
as bibliotecas e ambientes de programação nescessários no desenvolvimento do projeto, as aplicações
que servirão de auxílio, tanto no desenvolvimento do \emph{software} quanto no desenvolvimento do artigo,
e os computadores pessoais utilizados no desenvolvimento do trabalho.

\begin{itemize}
	\item Raspberry Pi.
	\item Módulo de camera para Raspberry Pi.
	\item Ambiente de programação com as linguagens C/C++ e Python.
	\item Biblioteca OpenCV de visão computacional.
	\item Sistema \LaTeX para documentação.
	\item Software de versionamento git.
	\item Ferramenta para reconhecimento ótico de caracteres Tesseract.
	\item Editor de texto vi
	\item IDE Clion e Pycharm.
	\item Dois computadores pessoais para propósito geral.
\end{itemize}


%\chapter{Levantamento de Requisitos}
%\input{levantamento_de_requisitos}

%\chapter{Desenvolvimento do Trabalho}
%\input{desenvolvimento}

%\chapter{Metodologia e Cronograma}
%\input{cronograma_de_atividades}

%\chapter{Comentários Finais}
%O reconhecimento automático de placas de carro mostrou-se ser uma área bastante
vasta, com diversos estudos sobre o tema e variadas técnicas para alcançar seu
objetivo. Dos trabalhos relacionados pesquisados não foi encontrado nenhum que
tenha sucesso em todas as suas tentativas, e os estudos comparativos levam a
crer que não há uma técnica que funcione bem para todos os casos. 

A solução desenvolvida nesse trabalho obteve baixa performance no computador
\emph{Raspberry Pi}, podendo ser vista apenas como um protótipo. O fato de ele
ser um computador de propósito geral contribuiu com essa baixa performance. Um
computador especializado em processamento de imagens, que possua uma placa de
video, teria resultados mais satisfatórios.

Com relação à quantidade de placas corretamente reconhecidas, o \emph{software}
também não teve uma taxa de acertos muito alta, ficando abaixo dos 50\%. O baixo
resultado se dá, principalmente, pela parte de extração da placa na imagem, onde
a maioria dos erros se concentra. Como as etapas do processamento são bem
distintas, é possível substituir a extração da placa por outras técnicas com
facilidade, podendo assim, facilmente avaliar outras estratégias.

Como trabalhos futuros fica sugerida a aplicação do software desenvolvido em
outros sistemas embarcados com maior poder de processamento que o
\emph{Raspberry Pi}. Um computador especializado no processamento de imagens
teria resultados melhores que um computador de propósito geral.

Outra maneira de combater o problema de performance que pode vir a ser estudado
seria uma maneira de analisar a imagem antes do processamento. Ao processar um
vídeo quadro a quadro, muitas imagens que são processadas não vão ter um bom
resultado devido a fatores externos e a posição do carro na imagem. Se for
possível fazer uma análise de alta performance na imagem, para avaliar se ela
está em boas condições para ser processada ou não, seria possível otimizar o
processo. 

Outro tema proposto como trabalho futuro, para melhorar a eficácia do
reconhecimento, seria otimização da técnica para poder extrair com mais
qualidade placas de carros em diferentes distâncias. Com essa melhoria é
possível analisar mais imagens em um vídeo de um carro que se locomove em
direção a câmera, podendo ter mais precisão no reconhecimento.


\bibliographystyle{abnt}
\bibliography{references}

%\begin{appendices}
%\input{apendices}
%\end{appendices}

\end{document}
