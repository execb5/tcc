%% LyX 1.3 created this file.  For more info, see http://www.lyx.org/.
%% Do not edit unless you really know what you are doing.
\documentclass[11pt,twoside,brazil]{pep}
\usepackage{a4wide}
\usepackage{geometry}
\geometry{verbose,a4paper}
\pagestyle{headings}
\usepackage{float}
\usepackage{multirow}
\usepackage[T1]{fontenc}
\usepackage[utf8x]{inputenc} 
%\usepackage[utf8]{inputenc}
\usepackage{graphicx}
%\usepackage[breaklinks=true]{hyperref}
%\usepackage{cite}
\usepackage{breakcites}
% \usepackage[latin1]{inputenc}
\usepackage{array}
\newcolumntype{P}[1]{>{\centering\arraybackslash}p{#1}}
%\makesavenoteenv{tabular}
%\makesavenoteenv{table}
\usepackage{tablefootnote}
\usepackage{scrextend}
\usepackage{hyperref}
\usepackage{multirow}
\usepackage[toc,page]{appendix}
\usepackage{enumerate}

\graphicspath{{images/}}

\makeatletter

%%%%%%%%%%%%%%%%%%%%%%%%%%%%%% LyX specific LaTeX commands.
%% Bold symbol macro for standard LaTeX users
\providecommand{\boldsymbol}[1]{\mbox{\boldmath $#1$}}

\floatstyle{ruled}
%\newfloat{algorithm}{tbp}{loa}
%\floatname{algorithm}{Algorithm}

%%%%%%%%%%%%%%%%%%%%%%%%%%%%%% Textclass specific LaTeX commands.
\usepackage{verbatim}

\usepackage{babel}
%\DeclareUnicodeCharacter{00A0}{-}
%\DeclareUnicodeCharacter{00A0}{~}
\makeatother

\setcounter{secnumdepth}{3}
\setcounter{tocdepth}{3}


\begin{document}

\title{Título}

 
\author{Daniel Antoniazzi Amarante e Matthias Oliveira Nunes}

\Advisor{Prof. Roland Teodorowitsch}

\maketitle

\maketitlerosto

%\begin{resumo}
%\input{resumo}
%\end{resumo}

%\begin{abstract}
%\input{abstract}
%\end{abstract}

\tableofcontents{}

\listoffigures

\listoftables

\chapter{Introdução}
\label{cha:intro}

O controle e identificação de veículos é usado nas mais diversas áreas, indo
desde serviços de pagamentos automatizados, como pedágios, até aplicações mais
críticas, como segurança de fronteiras e sistemas de vigilância de
tráfego~\cite{ahmad2015automatic}. E com o crescimento constante da frota de
carros no Brasil, aplicações para auxiliar neste trabalho tornam-se cada vez
mais nescessárias.  Com isso em mente, propomos neste trabalho uma solução de
aplicação embarcada de contagem de carros e reconhecimento de placas, visando a
crescente nescessidade de controle na área e as peculiaridades das placas
automotivas brasileiras, que nos impossibilitam de utilizar softwares
estrangeiros, nescessitando pesquisas locais neste tema.

O reconhecimento automático de placas de carros (ALPR) é a extração das informações
das placas de veículos a partir de uma imagem ou uma sequência de imagens, e a sua
utilização na vida real precisa de um processamento rápido e bem sucedido de placas 
sob diferentes condições ambientais, deve considerar as diferenças entre as placas
de diferentes nações, que terão cores, fontes, e linguas diferentes e se superar casos
onde as placas possam estar parcialmente cobertas com sujeira, luzes e acessórios dos 
carros.˜\cite{s2013automatic}

A solução proposta neste trabalho tem como diferencial o fato de permitir uma
análise e processamento das imagens em tempo real. Isso será realizado utilizando
um software embarcado, que estará coletando as imagens ao mesmo tempo que as analisa
e as envia para um servidor, permitindo que esses dados sejam úteis para uma tomada
de decisão imediata do usuário, por exemplo em uma aplicação de controle da lotação
de um estacionamento, onde ter as informações sobre os dados em tempo real é crucial
para a viabilidade do produto. Uma alternativa seria fazer a gravação das imagens em um 
momento, e o processamento em outro, entretanto, essa abordagem limitaria a utilidade do 
software.

Com o software será feita uma prova de conceito utilizando o estacionamento da
PUCRS, se for possível adquirir permissão, posicionando a camera em um ponto de
gargalo do estacionamento, que sirva de única entrada para uma determinada área,
com a qual será feita a contagem e identificação dos veículos, permitindo saber
qual a lotação daquele setor, e também identificar os veículos presentes.

Este trabalho está organizado da seguinte maneira: No capítulo~\ref{cha:trab}
serão apresentados trabalhos relacionados aos temas de reconhecimento de placas
e contagem de carros, no capítulo~\ref{cha:modelo} será apresentado o modelo
proposto, e nos capítulos~\ref{cha:crono}  e~\ref{cha:recursos} serão mostrados
o cronograma planejado e os recursos nescessários para a conclusão deste
trabalho respectivamente.



\chapter{Trabalhos Relacionados}
\label{cha:trab}

Uma solução para detectar e reconhecer placas de licenciamento brasileiras foi
proposta por Serro~\cite{serro2012deteccao} na PUCRS. Neste trabalho foram utilizadas
técnicas de segmentação de imagens, histograma, cisalhamento de imagens e reconhecimento
ótico de caracteres. Sua implementação tem uma série de restrições. É apenas considerado
um subconjunto de placas, excluindo placas de representação e de
deficientes auditivos. Não são reconhecidos estado e cidade da placa. E o
ambiente onde seria feita a foto para reconhecimento seria em situações onde o
veículo se encontra parado, com a câmera posicionada à frente do veículo.

A metodologia utilizada consistiu das seguintes etapas, calibração do sistema
para definir a região de interesse e o ângulo de cisalhamento, detecção da
placa, segmentação dos caracteres e aplicação do reconhecedor ótico de
caracteres.

Com a solução proposta foi obtida uma taxa de acerto de aproximadamente 54\%,
com tempo médio de execução de 0.062 segundos por imagem. A baixa taxa de acerto
pode ter se dada por problemas de foco e nitidez, o tamanho da placa em pixels
nas imagens e a existência de outros objetos em cena. Todos esses problemas
citados no desenvolvimento do projeto.

Em Ahmad et al.\cite{ahmad2015automatic} foi feito um estudo comparativo dos sistemas de
reconhecimento de placas automotivas automaticos. Segundo eles, o processo de
ler o conteúdo de uma placa passa por três estágios. O primeiro é a localização
ou extração da placa, que consiste no processo de localizar a placa do carro na
imagem. O segundo estágio é a separação dos caracteres, onde cada caractere
individual é separado dos outros para reconhecimento. E o terceiro e último
estágio é o reconhecimento do caractere em si, onde os caracteres extraídos da
imagem são identificados.

Neste trabalho implementados três diferentes métodos de localização de placa e dois
diferentes métodos de reconhecimento de caracteres, resultando em 6 diferentes
abordagens para o reconhecimento de placas. Todas essas combinações foram então
testadas contra diferentes conjuntos de dados.

Os resultados obtidos na experimentação não foram muito animadores, variando
entre 20 e 40 por cento. Um dos motivos para os maus resultados foi a variedade
de parametros nas imagens do conjunto de dados de teste. Tais parametros incluem
variações na distância, ângulo, iluminação e ambiente. Estes erros poderiam ser mitigados 
em sistemas reais com uma câmera com resolução fixa e de boa qualidade. A variação do
tamanho da placa afetou o desempenho de alguns algoritmos, mas em uma câmera
fixa é possível obter uma consistência e conseguir resultados mais aceitáveis.

Outros motivos para a baixa taxa de acerto foram a falta de pré-processamento
das imagens, que a análise não considerou, e a utilização de mais dados de
aprendizado para o reconhecimento ótico de caracteres.

Em Abtahi et al.~\cite{abtahi2015deep} foi feito um nova abordagem para a segmentação de
caracteres em imagens. De acordo com eles, o método padrão de segmentação
baseado em projeção sofre com variações consideráveis na região da placa envolto
aos caracteres. Estes autores propoem duas abordagens. A primeira é feita adaptando um método de
aprendizado por reforço, criando um agente que consiga achar os melhores caminhos para a segmentação. 
A segunda abordagem usa um método híbrido que utiliza a simplicidade e velocidade do método de projeção, 
mas com o poder do aprendizado por reforço.

De acordo com Wafy e Madbouly~\cite{wafy2016efficient}, o reconhecimento de uma placa consiste
em dois mecanismos principais: detecção de uma placa e em seguida a
identificação da mesma. O algoritmo proposto nesse artigo, faz os dois passos e
se baseia na distribuição semi-simétrica dos pontos de canto nas imagens de
carros e placas, e nas características morfológicas da região da placa. Essa
solução teve uma taxa de acerto de 97.5\% no processo de detecção e 92.8\% na
identificação, com o maior tempo de execução para um dos processos sendo de
0.3s. Com estes resultados seria possível utilizar este método em aplicações de tempo real.


\chapter{Modelo}
\label{cha:modelo}

O objetivo deste trabalho é desenvolver um software embarcado em um Raspberry Pi
equipado com módulo de câmera que seja capaz de reconhecer veículos, contá-los e
identificá-los baseado-se em sua placa. Todo o processamento deve acontecer em
tempo real com base nas imagens da câmera captadas no momento e a informação
coletada e processada deve ser enviada para um servidor que possa utilizar estes
dados.

Segundo~\cite{ahmad2015automatic}, o reconhecimento de placas automotivas requer
três passos, a localização da placa, a separação dos caracteres e o
reconhecimento dos caracteres. Para a localização da placa e separação dos
caracteres será utilizada a biblioteca OpenCV e a linguagem de programação, C,
C++ ou Python, pois são as linguagens mais usadas para OpenCV, a melhor
abordagem ainda será analisada. Para o reconhecimento dos caracteres será
estudada a possibilidade de utilizar o software Tesseract ou criar uma solução
própria. As escolhas a serem feitas tem como objetivo maximizar os resultados ao
final do trabalho, tentando criar um balanço entre facilidade de implementação e
qualidade do reconhecimento.



\chapter{Cronograma}
\label{cha:crono}

O desenvolvimento deste trabalho prevê as seguintes atividades:

\begin{table}[H]
\centering
\label{tab:crono}
\begin{tabular}{c|c|c|c}
    & Setembro & Outubro & Novembro \\ \hline
I   & X        &         &          \\
II  &          & X       &          \\
III &          & X       &          \\
IV  &          & X       &          \\
V   &          &         & X       
\end{tabular}
\caption{Cronograma proposto}
\end{table}

\begin{enumerate}[I]
	\item Pesquisar e estudar trabalhos relacionados atuais.
	\item Se familizarizar com a biblioteca OpenCV\@.
	\item Se familiarizar com a ferramenta para reconhecimento ótico de caracteres Tesseract.
	\item Se familiarizar com o Raspberry Pi e seu módulo de câmera.
	\item Definir modelo e iniciar implementação.
\end{enumerate}



\chapter{Recursos Necessários}
\label{cha:recursos}
Estão aqui discriminadas as peças físicas de \emph{hardware} que integrarão o
produto final desenvolvido, as bibliotecas e ambientes de programação
nescessários no desenvolvimento do projeto, as aplicações que servirão de
auxílio, tanto no desenvolvimento do \emph{software} quanto no desenvolvimento
do artigo, e os computadores pessoais utilizados no desenvolvimento do trabalho.
Para o desenvolvimento do trabalho serão nescessários os seguintes recursos:

\begin{itemize}
	\item Raspberry Pi.
	\item Módulo de câmera para Raspberry Pi.
	\item Ambiente de programação com as linguagens C/C++ e Python.
	\item Biblioteca OpenCV de visão computacional.
	\item Sistema \LaTeX para documentação.
	\item Software de versionamento git.
	\item Ferramenta para reconhecimento ótico de caracteres Tesseract.
	\item Editor de texto vi
	\item IDE Clion e Pycharm.
	\item Dois computadores pessoais para propósito geral.
\end{itemize}


%\chapter{Levantamento de Requisitos}
%\input{levantamento_de_requisitos}

%\chapter{Desenvolvimento do Trabalho}
%\input{desenvolvimento}

%\chapter{Metodologia e Cronograma}
%\input{cronograma_de_atividades}

%\chapter{Comentários Finais}
%O reconhecimento automático de placas de carro mostrou-se ser uma área bastante vasta, com bastante estudo a respeito e variadas técnicas para chegar no seu objetivo. Dos trabalhos relacionados pesquisados não foi encontrado nenhum que tenha sucesso em todas as suas tentativas, e os estudos comparativos levam a crer que não há uma técnica que funcione para todos os casos. 

A solução desenvolvida nesse trabalho obteve baixo desempenho no reconhecimento e na performance em sistema embarcado de propósito geral, podendo ser vista apenas como um protótipo. 

Como trabalhos relacionados fica sugerido a aplicação do software desenvolvido em outros sistemas embarcados com maior poder de processamento que o \emph{Raspberry Pi}. Um computador especializado no processamento de imagens teria resultados melhores que um computador de propósito geral.

Outra maneira de combater o problema de performance que pode vir a ser estudado seria uma maneira de analisar a imagem antes do processamento. Ao processar um vídeo quadro a quadro, muitas imagens que são processadas não vão ter um bom resultado, devido a fatores externos e a posição do carro na imagem. Uma trabalho que pode vir a ser feito seria descobrir como fazer uma análise de alta performance da imagem, para avaliar se ela está boa para ser processada ou não.

Outro tema proposto como trabalho futuro seria a da extração da placa para poder extrair com mais qualidade placas em carros em diferentes distancias. Com essa melhoria é possível analisar mais imagens em um vídeo de um carro que se locomove em direção a câmera, podendo ter mais precisão no reconhecimento.

\bibliographystyle{abnt}
\bibliography{references}

%\begin{appendices}
%\input{apendices}
%\end{appendices}

\end{document}
