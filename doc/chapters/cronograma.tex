O desenvolvimento deste trabalho prevê as seguintes atividades.
Estas atividades tem como objetivo definir o modelo que será implementado
e iniciar a implementação.

\begin{table}[H]
\centering
\label{tab:crono}
\begin{tabular}{c|c|c|c}
	& Setembro & Outubro & Novembro \\ \hline
I   & X        &         &          \\
II  &          & X       &          \\
III &          & X       &          \\
IV  &          & X       &          \\
V   &          &         & X       
\end{tabular}
\caption{Cronograma proposto}
\end{table}

\begin{ganttchart}[
    y unit title=0.5cm,
    y unit chart=0.6cm,
    time slot format=isodate-yearmonth,
    compress calendar,
    title/.append style={shape=rectangle, fill=black!10},
    title height=1,
    bar/.append style={fill=green!90},
    bar height=.6,
    bar label font=\normalsize\color{black!50},
    group top shift=.6,
    group height=.3,
    group peaks height=.2,
    bar incomplete/.append style={fill=green!40}
  ]{2016-07}{2017-07}
  \gantttitlecalendar{year} \\
  \gantttitlecalendar{month} \\
  \ganttset{progress label text={}}
  \ganttgroup{TCI: Projeto}{2016-08}{2016-11} \\
    \ganttbar[progress=00, name=i]{I}{2016-08}{2016-08} \\
    \ganttbar[progress=00, name=ii]{II}{2016-09}{2016-10} \\
    \ganttbar[progress=00, name=iii]{III}{2016-09}{2016-10} \\
    \ganttbar[progress=00, name=iv]{IV}{2016-09}{2016-10} \\
    \ganttbar[progress=00, name=v]{V}{2016-09}{2016-10} \\
    \ganttbar[progress=00, name=vi]{VI}{2016-09}{2016-10} \\
    \ganttbar[progress=00, name=vii]{VII}{2016-09}{2016-10} \\
    \ganttbar[progress=00, name=viii]{VIII}{2016-10}{2016-11} \\
  \ganttgroup{TCII: Implementação}{2017-03}{2017-06} \\
  % misc links
  \ganttset{progress label text={}}
  \ganttlink[link mid=0.25]{i}{viii}
  \ganttlink[link mid=0.25]{ii}{viii}
  \ganttlink[link mid=0.25]{iii}{viii}
  \ganttlink[link mid=0.25]{iv}{viii}
  \ganttlink[link mid=0.25]{v}{viii}
  \ganttlink[link mid=0.25]{vi}{viii}
  \ganttlink[link mid=0.25]{vii}{viii}
\end{ganttchart}

\begin{enumerate}[I]
	\item Pesquisar e estudar trabalhos relacionados atuais.
	\item Estudar a biblioteca OpenCV\@.
	\item Estudar a ferramenta para reconhecimento ótico de caracteres Tesseract.
	\item Estudar o Raspberry Pi e seu módulo de câmera.
	\item Definir modelo e iniciar implementação.
	\item Integrar o OpenCV e o Raspberry Pi e implementar uma prova de conceito que utilize os dois.
	\item Implementar uma aplicação que localize a placa de um carro em uma imagem utilizando OpenCV.
	\item Escrever o volume final de TCI.
\end{enumerate}

