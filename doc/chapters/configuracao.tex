O desenvolvimento deste trabalho requer o uso de \emph{hardware} e
\emph{software} existentes que nescessitam de um certo nivel de configuração
para serem utilizados concorrentemente.

\section{Raspberry Pi e módulo de câmera}
\label{sec:confrpi}

No \emph{Raspberry Pi} será instalado o sistema operacional \emph{Raspbian}. Sua
imagem pode ser baixada no site oficial,
\texttt{https://www.raspberrypi.org/downloads/raspbian/}. O módulo de câmera é
facilmente acoplado no \emph{Raspberry Pi} e para funcionar só é nescessário
habilitar a câmera na ferramenta de configuração do sistema que pode ser
acessada utilizando o comando \texttt{sudo raspi-config}. Para garantir que a
câmera está funcionando pode-se usar o comando \texttt{raspistill} para capturar
uma imagem.

\section{OpenCV}
\label{sec:confopencv}

A instalação do \emph{OpenCV} no \emph{Raspberry Pi} pode ser feita de diversas maneiras e customizações.
Para este trabalho optamos por baixar o código fonte em \texttt{https://github.com/opencv/opencv} e instalar
utilizando o \emph{software} \emph{CMake}. Para utilizar o \emph{OpenCV} com \emph{Python}, também é nescessário
instalar a biblioteca \emph{numpy}, utilizando o gerenciador de pacotes para programas em \emph{Python} \emph{pip}.

Uma vez instalado, pode-se acessar a biblioteca \emph{OpenCV} do código \emph{Python} importando o pacote \emph{cv2}.
Para integrar a camera do \emph{Raspberry Pi} com o \emph{OpenCV} também é recomendado instalar o pacote \emph{picamera},
mas não nescessário.

\section{Tesseract}
\label{sec:conftess}

A ferramenta \emph{Tesseract} pode ser instalada no \emph{Raspbian} utilizando o gerenciador de pacotes padrão do sistema,
utilizando o comando \texttt{sudo apt-get install tesseract-ocr}. Com isso é possível utilizar o \emph{Tesseract} na linha
de comando. Para utilizar com \emph{Python}, pode-se instalar o pacote \emph{pytesseract} com o gerenciador de pacotes \emph{pip},
que é um gerenciador de pacotes para programas em \emph{Python}. Para instalar o \emph{pytesseract} executa-se o comando
\texttt{sudo -H pip install pytesseract}.
