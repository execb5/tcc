\cite{WinNT} controle e identificação de veículos é usado nas mais diversas áreas, indo
desde serviços de pagamentos automatizados, como pedágios, até aplicações mais
críticas, como segurança de fronteiras, sistemas de vigilância de
tráfego~\cite{ahmad2015automatic} e sistema de busca por carros roubados.
Uma solução para a identificação de placas é, inclusive,  parte do Plano de Governo do atual prefeito eleito de Porto Alegre,
Nelson Marchezan. Ele pretende utilizar os sistemas de controle de velocidade da cidade para também
monitorar as placas de carros com o objetivo de identificar carros roubados~\cite{psdb2016marchezan}.

Com o crescimento constante da frota de carros no Brasil, aplicações para
auxiliar neste trabalho tornam-se cada vez
mais necessárias. Com isso em mente, propõe-se neste trabalho uma
solução de aplicação embarcada de reconhecimento de placas, visando
à crescente necessidade de controle na área. O fato de que há
peculiaridades nas placas automotivas
brasileiras, que impossibilitam a utilização de ferramentas configuradas
para placas estrangeiras, também evidencia a necessidade de soluções
locais para este problema.

O reconhecimento automático de placas de carros (\emph{Automatic License-Plate
Recognition}, ALPR) corresponde à extração
das informações das placas de veículos a partir
de uma imagem ou de uma sequência de imagens. A sua utilização na vida real
precisa de um processamento rápido e bem sucedido de placas sob diferentes
condições ambientais. Deve-se considerar as variações entre as placas de
diferentes nações, que terão cores, fontes, símbolos, padrões e línguas
diferentes. Também é preciso superar casos nos quais as placas possam estar
parcialmente cobertas com sujeira, luzes e acessórios dos
carros, e também a iluminação do ambiente e qualidade
da imagem adquirida~\cite{s2013automatic}.

O objetivo deste trabalho é desenvolver um \emph{software} embarcado em um \emph{Raspberry Pi}
(um computador de tamanho reduzido) equipado com módulo de câmera que seja
capaz de reconhecer veículos e identificá-los baseando-se em sua
placa. A solução proposta tem como diferencial o fato de permitir a
análise e processamento das imagens em tempo real. Isso é realizado utilizando um software embarcado, que coleta as imagens ao mesmo tempo
que as analisa. Essa abordagem evitaria ter que enviar a imagem inteira para ser processada em um servidor, ocupando menos largura de banda na transmissão. Dessa maneira é possível que esses dados sejam úteis para uma tomada de decisão imediata das "autoridades"~de controle de tráfego ou policiamento.

Um exemplo de uma aplicação, na qual a velocidade e a disponibilidade imediata das informações é crucial para a viabilidade do produto, seria em um \emph{software} de identificação de carros roubados. O sistema analisaria as placas dos carros que trafegam em uma rodovia e identificaria quais daqueles carros têm placas que correspondem a placas de carros roubados. Para que o \emph{software} seja eficiente é necessário que o processamento seja feito em tempo real. Qualquer atraso na identificação do veículo permite que ele se distancie muito, impedindo que as autoridades reajam a tempo.

Uma abordagem alternativa ao processamento em tempo real, seria fazer a captura das imagens em um momento, e o processamento em outro. Entretanto, desta maneira, a utilidade do \emph{software} é limitada, pois nas aplicações citadas, estes dados precisam estar disponíveis imediatamente.

A solução desenvolvida foi testada e analisada em duas partes: A primeira parte foi o teste de funcionamento, que tem como objetivo avaliar se o programa consegue identificar as placas, com qual frequência e em quais condições. A segunda foi o teste de performance, que tem como objetivo avaliar se o programa consegue identificar as placas em tempo aceitável para ser utilizado em situações reais. Assim foi possível aprimorar a solução para garantir melhor funcionamento.

Para o teste do funcionamento da detecção foram geradas múltiplas imagens, em condições e ângulos levemente diferentes. Com essas imagens foi possível analisar casos em que a detecção funciona e casos em que a detecção não funciona, para poder avaliar os motivos e fazer melhorias.

Para o teste da performance, foi posicionado o computador com a câmera em um ponto estratégico de um estacionamento para analisar os carros que passavam. O programa pode, assim, coletar informações referentes ao valor da placa dos carros que passavam por aquele ponto. Dessa maneira foi possível medir o desempenho do programa tanto na velocidade da execução quanto na quantidade de acertos obtidos, podendo assim, descobrir qual a viabilidade da implementação em casos reais.

Este trabalho foi organizado da seguinte maneira: no Capítulo~\ref{cha:fundamentacao}, serão mostrados e explicados os algoritmos e os termos técnicos utilizados no trabalho. No Capítulo~\ref{cha:trab},
serão apresentados trabalhos relacionados aos temas de reconhecimento de placas de carros. No Capítulo~\ref{cha:componentes}, serão apresentadas as tecnologias utilizadas e o ambiente de desenvolvimento. No Capítulo~\ref{cha:implementacao}, serão especificados os passos seguidos no desenvolvimento da aplicação, explicando cada etapa do projeto. No Capítulo~\ref{cha:resultados} serão apresentados os resultados práticos obtidos com este projeto. No Capítulo~\ref{cha:conclusao} será feita uma conclusão sobre o resultado do trabalho e serão sugeridos futuros trabalhos que podem surgir com base na pesquisa feita.
