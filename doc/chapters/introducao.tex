
O controle e identificação de veículos é usado nas mais diversas áreas, indo
desde serviços de pagamentos automatizados, como pedágios, até aplicações mais
críticas, como segurança de fronteiras e sistemas de vigilância de
tráfego~\cite{ahmad2015automatic}. E com o crescimento constante da frota de
carros no Brasil, aplicações para auxiliar neste trabalho tornam-se cada vez
mais nescessárias.  Com isso em mente, propomos neste trabalho uma solução de
aplicação embarcada de contagem de carros e reconhecimento de placas, visando a
crescente nescessidade de controle na área e as peculiaridades das placas
automotivas brasileiras, que nos impossibilitam de utilizar softwares
estrangeiros, nescessitando pesquisas locais neste tema.

A solução proposta neste trabalho tem como diferencial o fato de ser uma
aplicação embarcada, que faz a análise e processamento das imagens em tempo
real, e envia as informações para um servidor, para que esses dados possam ser
úteis para uma tomada de decisão imediata por parte dos usuários, como por
exemplo o controle de lotação de um estacionamento. Diferente da alternativa que
seria fazer a gravação das imagens em um primeiro momento para fazer a análise
em uma outra ocasião, limitando a utilidade desses dados.

Com o software será feita uma prova de conceito utilizando o estacionamento da
PUCRS, se for possível adquirir permissão, posicionando a camera em um ponto de
gargalo do estacionamento, que sirva de única entrada para uma determinada área,
com a qual será feita a contagem e identificação dos veículos, permitindo saber
qual a lotação daquele setor, e também identificar os veículos presentes.

Este trabalho está organizado da seguinte maneira: No capítulo~\ref{cha:trab}
serão apresentados trabalhos relacionados aos temas de reconhecimento de placas
e contagem de carros, no capítulo~\ref{cha:modelo} será apresentado o modelo
proposto, e nos capítulos~\ref{cha:crono}  e~\ref{cha:recursos} serão mostrados
o cronograma planejado e os recursos nescessários para a conclusão deste
trabalho respectivamente.






