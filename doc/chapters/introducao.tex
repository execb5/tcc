O controle e identificação de veículos é usado nas mais diversas áreas, indo
desde serviços de pagamentos automatizados, como pedágios, até aplicações mais
críticas, como segurança de fronteiras, sistemas de vigilância de
tráfego~\cite{ahmad2015automatic} e sistema de busca por carros roubados.
Uma solução para identificar é, inclusive,  parte do plano de governo do atual prefeito eleito de Porto Alegre
Nelson Marchezan. Ele pretende utilizar os sistemas de controle de velocidade da cidade para também
monitorar as placas de carros com o objetivo de identificar carros roubados~\cite{psdb2016marchezan}.
Com o crescimento constante da frota de carros no Brasil, aplicações para
auxiliar neste trabalho tornam-se cada vez
mais nescessárias. Com isso em mente, propõe-se neste trabalho uma solução de
aplicação embarcada de reconhecimento de placas, visando a
crescente nescessidade de controle na área e as peculiaridades das placas
automotivas brasileiras, que nos impossibilitam de utilizar ferramentas
configuradas para placas estrangeiras, nescessitando pesquisas locais neste
tema.

O reconhecimento automático de placas de carros (\emph{Automatic License-Plate
Recognition}, ALPR) é a extração das informações das placas de veículos a partir
de uma imagem ou de uma sequência de imagens. A sua utilização na vida real
precisa de um processamento rápido e bem sucedido de placas sob diferentes
condições ambientais. Deve-se considerar as diferenças entre as placas de
diferentes nações, que terão cores, fontes, símbolos, padrões e línguas
diferentes. Também é preciso superar casos onde as placas possam estar
parcialmente cobertas com sujeira, luzes e acessórios dos
carros, e também a iluminação do ambiente e qualidade
da imagem adquirida.~\cite{s2013automatic}

A solução proposta neste trabalho tem como diferencial o fato de permitir uma
análise e processamento das imagens em tempo real. Isso será realizado
utilizando um software embarcado, que estará coletando as imagens ao mesmo tempo
que as analisa e as envia para um servidor. Com essa abordagem, será possível
que esses dados sejam úteis para uma tomada de decisão imediata do usuário.

Um exemplo de uma aplicação, onde a velocidade e a disponibilidade imediata das informações é
crucial para a viabilidade do produto, seria em um \emph{software} de identificação
de carros roubados. O sistema analisaria as placas dos carros que trafegam em uma
rodovia e identificaria quais daqueles carros tem placas que correspondem a placas
de carros roubados. Para que o \emph{software} seja eficiente é nescessário
que o processamento seja feito em tempo real, qualquer atraso na identificação do
veículo permite que ele se distancie muito, impedindo que as autoridades reajam a tempo.

Uma abordagem alternativa seria fazer a gravação das imagens em um momento, e o
processamento em outro. Entretanto, desta maneira, a utilidade do
\emph{software} seria limitada, pois na maioria das aplicações reais estes dados
precisam estar disponíveis imediatamente.

Com o resultado da implementação proposta será feita uma prova de conceito
utilizando o estacionamento da PUCRS, se for possível adquirir permissão.
Posicionando o computador com a câmera em um ponto estratégico do
estacionamento, serão analisados os carros que passarem. O programa então
coletará informações referentes ao número de carros que passaram por aquele ponto
e suas respectivas placas.

Este trabalho está organizado da seguinte maneira: No capítulo~\ref{cha:trab},
serão apresentados trabalhos relacionados aos temas de reconhecimento de placas
e contagem de carros. No capítulo~\ref{cha:modelo}, será apresentado o modelo
proposto. No capítulo~\ref{cha:implementacao}, será especificado os passos a serem
seguidos no desenvolvimento da aplicação, classificando os algoritmos a serem utilizados
em cada etapa. No capítulo~\ref{cha:configuracao}, será demonstrada como deve ser feita
a configuração e instalação das ferramentas nescessárias para o desenvolvimento do trabalho.
 E nos capítulos~\ref{cha:crono}  e~\ref{cha:recursos}, serão mostrados
o cronograma planejado e os recursos nescessários para a conclusão deste
trabalho respectivamente.
