
O controle e identificação de veículos é usado nas mais diversas áreas, indo
desde serviços de pagamentos automatizados, como pedágios, até aplicações mais
críticas, como segurança de fronteiras e sistemas de vigilância de
tráfego~\cite{ahmad2015automatic}. E com o crescimento constante da frota de
carros no Brasil, aplicações para auxiliar neste trabalho tornam-se cada vez
mais nescessárias.  Com isso em mente, propomos neste trabalho uma solução de
aplicação embarcada de contagem de carros e reconhecimento de placas, visando a
crescente nescessidade de controle na área e as peculiaridades das placas
automotivas brasileiras, que nos impossibilitam de utilizar softwares
estrangeiros, nescessitando pesquisas locais neste tema.

O reconhecimento automático de placas de carros (ALPR) é a extração das
informações das placas de veículos a partir de uma imagem ou uma sequência de
imagens, e a sua utilização na vida real precisa de um processamento rápido e
bem sucedido de placas sob diferentes condições ambientais, deve considerar as
diferenças entre as placas de diferentes nações, que terão cores, fontes, e
linguas diferentes e se superar casos onde as placas possam estar parcialmente
cobertas com sujeira, luzes e acessórios dos carros.~\cite{s2013automatic}

A solução proposta neste trabalho tem como diferencial o fato de permitir uma
análise e processamento das imagens em tempo real. Isso será realizado
utilizando um software embarcado, que estará coletando as imagens ao mesmo tempo
que as analisa e as envia para um servidor, permitindo que esses dados sejam
úteis para uma tomada de decisão imediata do usuário, por exemplo em uma
aplicação de controle da lotação de um estacionamento, onde ter as informações
sobre os dados em tempo real é crucial para a viabilidade do produto. Uma
alternativa seria fazer a gravação das imagens em um momento, e o processamento
em outro, entretanto, essa abordagem limitaria a utilidade do software.

Com o software será feita uma prova de conceito utilizando o estacionamento da
PUCRS, se for possível adquirir permissão, posicionando a camera em um ponto de
gargalo do estacionamento, que sirva de única entrada para uma determinada área,
com a qual será feita a contagem e identificação dos veículos, permitindo saber
qual a lotação daquele setor, e também identificar os veículos presentes.

Este trabalho está organizado da seguinte maneira: No capítulo~\ref{cha:trab}
serão apresentados trabalhos relacionados aos temas de reconhecimento de placas
e contagem de carros, no capítulo~\ref{cha:modelo} será apresentado o modelo
proposto, e nos capítulos~\ref{cha:crono}  e~\ref{cha:recursos} serão mostrados
o cronograma planejado e os recursos nescessários para a conclusão deste
trabalho respectivamente.

