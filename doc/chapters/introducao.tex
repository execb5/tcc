
O controle e identificação de veículos é usado nas mais diversas áreas, indo
desde serviços de pagamentos automatizados, como pedágios, até aplicações mais
críticas, como segurança de fronteiras e sistemas de vigilância de
tráfego~\cite{ahmad2015automatic}. E com o crescimento constante da frota de
carros no Brasil, aplicações para auxiliar neste trabalho tornam-se cada vez
mais nescessárias. Com isso em mente, propõe-se neste trabalho uma solução de
aplicação embarcada de contagem de carros e reconhecimento de placas, visando a
crescente nescessidade de controle na área e as peculiaridades das placas
automotivas brasileiras, que nos impossibilitam de utilizar ferramentas
configuradas para placas estrangeiras, nescessitando pesquisas locais neste
tema.

O reconhecimento automático de placas de carros (\emph{Automatic license-plate
recognition}, ALPR) é a extração das informações das placas de veículos a partir
de uma imagem ou de uma sequência de imagens. A sua utilização na vida real
precisa de um processamento rápido e bem sucedido de placas sob diferentes
condições ambientais. Deve-se considerar as diferenças entre as placas de
diferentes nações, que terão cores, fontes, símbolos, padrões e linguas
diferentes. Também é preciso superar casos onde as placas possam estar
parcialmente cobertas com sujeira, luzes e acessórios dos
carros.~\cite{s2013automatic}

A solução proposta neste trabalho tem como diferencial o fato de permitir uma
análise e processamento das imagens em tempo real. Isso será realizado
utilizando um software embarcado, que estará coletando as imagens ao mesmo tempo
que as analisa e as envia para um servidor. Com essa abordagem, será possível
que esses dados sejam úteis para uma tomada de decisão imediata do usuário. Um
exemplo de uma aplicação onde a velocidade e a imediatidade das informações é
crucial para a viabilidade do produto seria em um \emph{software} de controle de
lotação de estacionamento. O usuário que utilizaria uma ferramenta para
descobrir onde se encontram as vagas para estacionar precisa que estes dados
sejam atualizados em tempo real, correndo o risco de a vaga denunciada pela
aplicação não estar mais disponivel de fato, se o processamento não for ao vivo.

Uma abordagem alternativa seria fazer a gravação das imagens em um momento, e o
processamento em outro. Entretanto, desta maneira, a utilidade do
\emph{software} seria limitada, pois na maioria das aplicações reais estes dados
precisam estar disponíveis imediatamente.

Com o resultado da implementação proposta será feita uma prova de conceito
utilizando o estacionamento da PUCRS, se for possível adquirir permissão.
Posicionando o computador com a câmera em um ponto estratégico do
estacionamento, serão analisados os carros que passarem. Esta localização deverá
ser um ponto de gargalo do estacionamento, que sirva de única entrada para uma
determinada área. Fazendo a contagem dos carros que passarem por este caminho,
será possível obter informações sobre a quantidade de carros que se encontram
naquela área e identificar os veículos presentes. Com essas informações à
disposição, seria possível alimentar uma aplicação móvel que informe o usuário
quais os melhores lugares para encontrar vagas para estacionar, por exemplo.

Este trabalho está organizado da seguinte maneira: No capítulo~\ref{cha:trab},
serão apresentados trabalhos relacionados aos temas de reconhecimento de placas
e contagem de carros. No capítulo~\ref{cha:modelo}, será apresentado o modelo
proposto. E nos capítulos~\ref{cha:crono}  e~\ref{cha:recursos}, serão mostrados
o cronograma planejado e os recursos nescessários para a conclusão deste
trabalho respectivamente.

