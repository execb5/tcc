O reconhecimento automático de placas de carro mostrou-se ser uma área bastante
vasta, com diversos estudos sobre o tema e variadas técnicas para alcançar seu
objetivo. Dos trabalhos relacionados pesquisados não foi encontrado nenhum que
tenha sucesso em todas as suas tentativas, e os estudos comparativos levam a
crer que não há uma técnica que funcione bem para todos os casos. 

A solução desenvolvida nesse trabalho obteve baixa performance no computador
\emph{Raspberry Pi}, podendo ser vista apenas como um protótipo. O fato de ele
ser um computador de propósito geral contribuiu com essa baixa performance. Um
computador especializado em processamento de imagens, que possua uma placa de
video, teria resultados mais satisfatórios.

Com relação à quantidade de placas corretamente reconhecidas, o \emph{software}
também não teve uma taxa de acertos muito alta, ficando abaixo dos 50\%. O baixo
resultado se dá, principalmente, pela parte de extração da placa na imagem, onde
a maioria dos erros se concentra. Como as etapas do processamento são bem
distintas, é possível substituir a extração da placa por outras técnicas com
facilidade, podendo assim, facilmente avaliar outras estratégias.

Como trabalhos futuros fica sugerida a aplicação do software desenvolvido em
outros sistemas embarcados com maior poder de processamento que o
\emph{Raspberry Pi}. Um computador especializado no processamento de imagens
teria resultados melhores que um computador de propósito geral.

Outra maneira de combater o problema de performance que pode vir a ser estudado
seria uma maneira de analisar a imagem antes do processamento. Ao processar um
vídeo quadro a quadro, muitas imagens que são processadas não vão ter um bom
resultado devido a fatores externos e a posição do carro na imagem. Se for
possível fazer uma análise de alta performance na imagem, para avaliar se ela
está em boas condições para ser processada ou não, seria possível otimizar o
processo. 

Outro tema proposto como trabalho futuro, para melhorar a eficácia do
reconhecimento, seria otimização da técnica para poder extrair com mais
qualidade placas de carros em diferentes distâncias. Com essa melhoria é
possível analisar mais imagens em um vídeo de um carro que se locomove em
direção a câmera, podendo ter mais precisão no reconhecimento.
