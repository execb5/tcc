O reconhecimento automático de placas de carro mostrou-se ser uma área bastante vasta, com bastante estudo a respeito e variadas técnicas para chegar no seu objetivo. Dos trabalhos relacionados pesquisados não foi encontrado nenhum que tenha sucesso em todas as suas tentativas, e os estudos comparativos levam a crer que não há uma técnica que funcione para todos os casos. 

A solução desenvolvida nesse trabalho obteve baixo desempenho no reconhecimento e na performance em sistema embarcado de propósito geral, podendo ser vista apenas como um protótipo. 

Como trabalhos relacionados fica sugerido a aplicação do software desenvolvido em outros sistemas embarcados com maior poder de processamento que o \emph{Raspberry Pi}. Um computador especializado no processamento de imagens teria resultados melhores que um computador de propósito geral.

Outra maneira de combater o problema de performance que pode vir a ser estudado seria uma maneira de analisar a imagem antes do processamento. Ao processar um vídeo quadro a quadro, muitas imagens que são processadas não vão ter um bom resultado, devido a fatores externos e a posição do carro na imagem. Uma trabalho que pode vir a ser feito seria descobrir como fazer uma análise de alta performance da imagem, para avaliar se ela está boa para ser processada ou não.

Outro tema proposto como trabalho futuro seria a da extração da placa para poder extrair com mais qualidade placas em carros em diferentes distancias. Com essa melhoria é possível analisar mais imagens em um vídeo de um carro que se locomove em direção a câmera, podendo ter mais precisão no reconhecimento.