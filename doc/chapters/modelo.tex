
O objetivo deste trabalho é desenvolver um software embarcado em um Raspberry Pi
equipado com módulo de câmera que seja capaz de reconhecer veículos, contá-los e
identificá-los baseado-se em sua placa. Todo o processamento deve acontecer em
tempo real com base nas imagens da câmera captadas no momento e a informação
coletada e processada deve ser enviada para um servidor que possa utilizar estes
dados.

O fluxo esperado do software é o seguinte. O computador Raspberry pi ficará postado
em uma rota de fluxo frequente de carros, posicionado de maneira que consiga capturar
imagens das placas em boa qualidade com sua camera. Será nescessário o módulo de câmera
e de bateria, ou haver uma fonte de energia por perto. Ao capturar as imagens, o computador
irá processá-las localmente, seguindo os passos do reconhecimento de placas e contagem de carros.
Após obtida, a informação será enviada para um servidor simples que irá armazená-la. Para o envio
das informações, o computador também precisará estar equipado com algum módulo de internet.

Segundo~\cite{ahmad2015automatic}, o reconhecimento de placas automotivas requer
três passos, a localização da placa, a separação dos caracteres e o
reconhecimento dos caracteres. Para a localização da placa e separação dos
caracteres será utilizada a biblioteca OpenCV e a linguagem de programação C,
C++ ou Python, pois são as linguagens mais usadas para OpenCV, a melhor
abordagem ainda será analisada. Para o reconhecimento dos caracteres será
estudada a possibilidade de utilizar o software Tesseract ou criar uma solução
própria. As escolhas a serem feitas tem como objetivo maximizar os resultados ao
final do trabalho, tentando criar um balanço entre facilidade de implementação e
qualidade do reconhecimento.

O trabalho será realizado em etapas, tendo como objetivo intermediário conseguir
fazer o reconhecimento e a contagem dos carros, e como objetivo final a
identificação das placas.

\section{Raspberry Pi}
\label{sec:raspi}

Raspberry Pi é um computador construído em uma placa de circuito do tamanho de
um cartão de crédito desenvolvido pela Raspberry Pi
Foundation\footnote{https://www.raspberrypi.org/}.

\section{OpenCV}
\label{sec:opencv}

OpenCV (Open Source Computer Vision Library) é uma biblioteca open source de
visão computacional e aprendizado de máquina. Contém mais de 2500 algoritmos
otimizados nessas áreas, incluindo algoritmos clássicos e recentes. A biblioteca
é escrita nativamente em C++, e dispõe de interfaces para C, C++, Python, Java e
MATLAB, suportando os sistemas operacionais Windows, Linux, Android e Mac
OS.\footnote{http://opencv.org/}

\section{OCR}
\label{sec:ocr}

Reconhecimento Ótico de Caracteres(OCR) consiste da conversão de textos em
formato de imagem para o formato reconhecido por maquina. É o método mais
eficiente para fazer o processamento de imagem para
texto.~\cite{mohit2015designing}

Uma ferramenta conhecida de OCR é o
Tesseract\footnote{https://github.com/tesseract-ocr/tesseract}. É uma ferramenta
open source de reconhecimento ótico de caracteres que suporta múltiplas linguas.
É essencialmente um algoritmo de comparação de templates, e as amostras de
caracteres podem ser auto-treinados.~\cite{ho2016intelligent}

\section{Placa de Transito Brasileira}

Segundo o código de transito brasileiro~\cite{brasil1997lei}, todos os veículos são 
identificados por meio de placas dianteira e traseira. 
Elas são identificadas por uma tarja na parte superior contendo a sigla do estado e o nome do município, e pelo código 
de identificação unico, composto por três letras, seguidas por quatro digitos, separados por um hífen. Veículos particulares,
de aluguel, oficial, de experiência, de aprendizagem e de fabricante tem suas dimensões de 130mmx400mm e altura dos
caracteres de 63mm, caso a placa não caiba no receptáculo ela pode ser reduzida em até 15\%. As placas de motocicleta,
motoneta, ciclomotor e triciclos autorizados tem dimensões de 136mmx187mm e altura de caracteres de 42mm. Imagem das placas
com suas dimensões podem ser vistos na Figura 1. A tipologia dos caracteres das placas utiliza a fonte Mandatory Figura 2,
e as placas de categorias diferentes de veículos são diferenciadas pelas suas cores Tabela 1.

\begin{table}[]
\centering
\caption{My caption}
\label{my-label}
\begin{tabular}{|l|l|l|}
\hline
\textbf{Categoria do Veículo}                                               & \textbf{Cor de Fundo} & \textbf{Cor de Caracteres} \\ \hline
Particular                                                                  & Cinza                 & Preto                      \\ \hline
Aluguel                                                                     & Vermelho              & BRanco                     \\ \hline
Experiência/Fabricante                                                      & Verde                 & Branco                     \\ \hline
Aprendizagem                                                                & Branco                & Vermelho                   \\ \hline
Coleção                                                                     & Preto                 & Cinza                      \\ \hline
Oficial                                                                     & Branco                & Preto                      \\ \hline
Missão Diplomática                                                          & Azul                  & Branco                     \\ \hline
Corpo Consular                                                              & Azul                  & Branco                     \\ \hline
Organismo Internacional                                                     & Azul                  & Branco                     \\ \hline
Corpo Diplomático                                                           & Azul                  & Branco                     \\ \hline
\begin{tabular}[c]{@{}l@{}}Organismo Consular/\\ Internacional\end{tabular} & Azul                  & Branco                     \\ \hline
\begin{tabular}[c]{@{}l@{}}Acordo Cooperação\\ Internacional\end{tabular}   & Azul                  & Branco                     \\ \hline
Representação                                                               & Preto                 & Dourado                    \\ \hline
\end{tabular}
\end{table}
