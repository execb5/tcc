
O objetivo deste trabalho é desenvolver um software embarcado em um Raspberry Pi
equipado com módulo de câmera que seja capaz de reconhecer veículos, contá-los e
identificá-los baseado-se em sua placa. Todo o processamento deve acontecer em
tempo real com base nas imagens da câmera captadas no momento e a informação
coletada e processada deve ser enviada para um servidor que possa utilizar estes
dados.

Segundo~\cite{ahmad2015automatic}, o reconhecimento de placas automotivas requer
três passos, a localização da placa, a separação dos caracteres e o
reconhecimento dos caracteres. Para a localização da placa e separação dos
caracteres será utilizada a biblioteca OpenCV e a linguagem de programação, C,
C++ ou Python, pois são as linguagens mais usadas para OpenCV, a melhor
abordagem ainda será analisada. Para o reconhecimento dos caracteres será
estudada a possibilidade de utilizar o software Tesseract ou criar uma solução
própria. As escolhas a serem feitas tem como objetivo maximizar os resultados ao
final do trabalho, tentando criar um balanço entre facilidade de implementação e
qualidade do reconhecimento.

