
O objetivo deste trabalho é desenvolver um software embarcado em um Raspberry Pi
equipado com módulo de câmera que seja capaz de reconhecer veículos, contá-los e
identificá-los baseado-se em sua placa. Todo o processamento deve acontecer em
tempo real com base nas imagens da câmera captadas no momento e a informação
coletada e processada deve ser enviada para um servidor que possa utilizar estes
dados.

Segundo~\cite{ahmad2015automatic}, o reconhecimento de placas automotivas requer
três passos, a localização da placa, a separação dos caracteres e o
reconhecimento dos caracteres. Para a localização da placa e separação dos
caracteres será utilizada a biblioteca OpenCV e a linguagem de programação C,
C++ ou Python, pois são as linguagens mais usadas para OpenCV, a melhor
abordagem ainda será analisada. Para o reconhecimento dos caracteres será
estudada a possibilidade de utilizar o software Tesseract ou criar uma solução
própria. As escolhas a serem feitas tem como objetivo maximizar os resultados ao
final do trabalho, tentando criar um balanço entre facilidade de implementação e
qualidade do reconhecimento.

O trabalho será realizado em etapas, tendo como objetivo intermediário conseguir 
fazer o reconhecimento e a contagem dos carros, e como objetivo final a identificação
das placas.

\subsection{Raspberry Pi}

Raspberry Pi é um computador construído em uma placa de circuito do tamanho de um cartão 
de crédito desenvolvido pela Raspberry Pi Foundation.\footnote{https://www.raspberrypi.org/}

\subsection{OpenCV}

OpenCV (Open Source Computer Vision Library) é uma biblioteca open source de visão 
computacional e aprendizado de máquina. Contém mais de 2500 algoritmos otimizados nessas
áreas, incluindo algoritmos clássicos e recentes. A biblioteca é escrita nativamente
em C++, e dispõe de interfaces para C, C++, Python, Java e MATLAB, suportando os sistemas 
operacionais Windows, Linux, Android e Mac OS. \footnote{http://opencv.org/}

\subsection{OCR}

Reconhecimento Ótico de Caracteres(OCR) consiste da conversão de textos em formato de imagem para
o formato reconhecido por maquina. É o método mais eficiente para fazer o processamento
de imagem para texto.~\cite{mohit2015designing}

Uma ferramenta conhecida de OCR é o Tesseract\footnote{https://github.com/tesseract-ocr/tesseract}.
É uma ferramenta open source de reconhecimento ótico de caracteres que suporta múltiplas
linguas. É essencialmente um algoritmo de comparação de templates, e as amostras de caracteres podem
ser auto-treinados.~\cite{ho2016intelligent}
