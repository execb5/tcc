Uma solução para detectar e reconhecer placas de licenciamento brasileiras foi
proposta por Serro~\cite{serro2012deteccao} na PUCRS\@. Neste trabalho foram
utilizadas técnicas de segmentação de imagens, histograma, cisalhamento de
imagens e reconhecimento ótico de caracteres. A metodologia utilizada consistiu das
seguintes etapas, calibração do sistema
para definir a região de interesse e o ângulo de cisalhamento, detecção da
placa, segmentação dos caracteres e aplicação do reconhecedor ótico de
caracteres.

Com a solução proposta por Serro~\cite{serro2012deteccao}  foi obtida uma taxa de
acerto de aproximadamente 54\%, com tempo médio de execução de 0,062 segundos por imagem.
A baixa taxa de acerto pode ter sido por problemas de foco e nitidez, o tamanho da placa em
\emph{pixels} nas imagens e a existência de outros objetos em cena. Todos esses
problemas foram citados no desenvolvimento do projeto.

Em Ahmad et al.~\cite{ahmad2015automatic} foi feito um estudo comparativo dos
sistemas de reconhecimento de placas automotivas automáticos. Segundo estes
autores, o processo de ler o conteúdo de uma placa passa por três estágios. O
primeiro é a localização ou extração da placa, que consiste no processo de
localizar a placa do carro na imagem. O segundo estágio é a separação dos
caracteres, onde cada caractere individual é separado dos outros para
reconhecimento. E o terceiro e último estágio é o reconhecimento do caractere em
si, onde os caracteres extraídos da imagem são identificados.

Neste trabalho foram implementados três diferentes métodos de localização de
placa e dois diferentes métodos de reconhecimento de caracteres, resultando em 6
diferentes abordagens para o reconhecimento de placas. Todas essas combinações
foram então testadas contra diferentes conjuntos de dados.

Os resultados obtidos por Ahmad et al.~\cite{ahmad2015automatic} na experimentação não
foram muito animadores, variando entre 20 e 40 por cento.
Um dos motivos para os maus resultados foi a variedade de parâmetros nas imagens do conjunto
de dados de teste. Tais parâmetros incluem
variações na distância, ângulo, iluminação e ambiente. Acreditamos que estes erros poderiam ser
mitigados em sistemas reais com uma câmera com resolução fixa e de boa
qualidade. A variação do tamanho da placa afetou o desempenho de alguns
algoritmos, mas em uma câmera fixa é possível obter uma consistência e conseguir
resultados mais aceitáveis.

Outros motivos para a baixa taxa de acerto foram a falta de pré-processamento
das imagens, que a análise não considerou, e a utilização de mais dados de
aprendizado para o reconhecimento ótico de caracteres.

Em Abtahi et al.~\cite{abtahi2015deep} foram feitas novas abordagens para a
segmentação de caracteres em imagens. De acordo com eles, o método padrão de
segmentação baseado em projeção sofre com variações consideráveis na região da
placa ao redor dos caracteres, portanto estes autores propuseram duas abordagens.
A primeira é feita adaptando um método de aprendizado por reforço, criando um agente que
consiga achar os melhores caminhos para a segmentação.  A segunda abordagem usa
um método híbrido que utiliza a simplicidade e velocidade do método de projeção,
mas com o poder do aprendizado por reforço.

De acordo com Wafy e Madbouly~\cite{wafy2016efficient}, o reconhecimento de uma
placa consiste em dois mecanismos principais: detecção de uma placa e em seguida
a sua identificação. O algoritmo proposto nesse artigo, faz os dois passos e se
baseia na distribuição semi-simétrica dos pontos de canto nas imagens de carros
e placas, e nas características morfológicas da região da placa. Essa solução
teve uma taxa de acerto de 97,5\% no processo de detecção e 92,8\% na
identificação, com o maior tempo de execução para um dos processos sendo de
0,3s. Com estes resultados seria possível utilizar este método em aplicações de
tempo real.

Com relação ao uso de sistema embarcado para executar o reconhecimento, Arth et
al.~\cite{arth2007real} trabalharam no desenvolvimento de um sistema de reconhecimento
de placas de carro em um processador de sinal digital (\emph{Digital Signal Processor}, DSP).
\emph{DSP} são microprocessadores especializados em
processar sinais digitais como áudio ou vídeo em tempo real.~\cite{yovits1993advances} O processador utilizado
neste trabalho específico foi um \emph{Texas Instruments C64} com \emph{1MB} de \emph{cache RAM} e um outro bloco
de memória mais lento, \emph{SDRAM} de \emph{16MB}. O processador não possui camera integrada mas permite a conexão de
uma fonte de vídeo analógica ou digital. Na solução implementada foi utilizada uma camera com resolução de \emph{352x288 pixels}.

Com sua implementação, Arth et al.~\cite{arth2007real} foram capazes de conseguir localizar a placa em
\emph{7.30 ms}, levando mais aproximadamente \emph{1 ms} para identificar cada caractere. Não é informado, no artigo,
a taxa de sucesso de cada reconhecimento. Os autores ainda concluem que por o tempo de detecção da placa ser superior
ao tempo de reconhecimento dos caracteres, e que este algoritmo deve ser melhorado.

Analisando os trabalhos feitos na área, nota-se que por mais que os algoritmos variem,
a base da detecção de placas permanece parecida. Eles costumam ser divididos em pelo
menos duas partes, a localização da placa e a detecção dos caracteres. Inclusive, Ahmad et al~\cite{ahmad2015automatic}
misturou diferentes algoritmos, utilizando a localização de um e a detecção de
outro, demonstrando que os dois passos são bem independentes.

Pode-se notar que o reconhecimento de placas de carros é uma área bastante estudada,
com diversas abordagens diferentes e grande variação de resultados. Entretanto,
um problema que ainda existe, é a grande diferença nas placas de diferentes países,
no estilo, fonte, caracteres utilizados e padrão do texto. A solução deste problema
é a criação de uma solução local de reconhecimento de placas.

O \emph{software open source Openalpr}\footnote{https://github.com/openalpr/openalpr}
criou uma solução de reconhecimento de placas que permite que a comunidade contribua
treinando o reconhecedor de caracteres a reconhecer as placas de seus países para contornar
esse problema.
