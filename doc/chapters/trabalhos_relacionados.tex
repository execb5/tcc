
Uma solução para detectar e reconhecer placas de licenciamento brasileiras foi
proposta por~\cite{serro2012deteccao} na PUCRS, utilizando técnicas de
segmentação de imagens, histograma, cisalhamento de imagens e reconhecimento
ótico de caracteres. Sua implementação dem uma série de restrições, são apenas
considerados um subconjunto de placas, excluindo placas de representação e de
deficientes auditivos, não sao reconhecidos estado e cidade da placa, e o
ambiente onde seria feita a foto para reconhecimento seria em situações onde o
veículo se encontra parado, com a camera posicionada a frente do veículo.

A metodologia utilizada consistiu das seguintes etapas, calibração do sistema
para deifnir a região de interesse e o ângulo de cisalhamento, detecção da
placa, segmentação dos caracteres e aplicação do reconhecedor ótico de
caracteres.

Com a solução proposta foi obtida uma taxa de acerto de aproximadamente 54\%,
com tempo médio de execução de 0.062 segundos por imagem. A baixa taxa de acerto
pode ter se dada por problemas de foco e nitidez, o tamanho da placa em pixels
nas imagens e a existência de outros objetos em cena, que foram problemas
citados no desenvolvimento do projeto.

Em \cite{ahmad2015automatic} foi feito um estudo comparativo dos sistemas de
reconhecimento de placas automotivas automaticos, segundo eles, o processo de
ler o conteúdo de uma placa passa por três estágios. O primeiro é a localização
ou extração da placa, que consiste no processo de localizar a placa do carro na
imagem. O segundo estágio é a separação dos caracteres, onde cada caracter
individual é separado dos outros para reconhecimento. E o terceiro e último
estágio é o reconhecimento do caracter em si, onde os caracteres extraídos da
imagem são identificados.

Foram implementados três diferentes métodos de localização de placa e dois
diferentes métodos de reconhecimento de caracteres, resultando em 6 diferentes
abordagens para o reconhecimento de placas.



