Segundo Ahmad et al.~\cite{ahmad2015automatic}, o reconhecimento de placas
automotivas requer três passos, a localização da placa, a separação dos
caracteres e o reconhecimento dos caracteres.~\cite{s2013automatic} ainda defende que são nescessários os mesmos passos, com a inclusão da aquisição das imagens como passo inicial. Para a localização da placa e
separação dos caracteres serão utilizados a biblioteca OpenCV e a linguagem de
programação C, C++ ou Python. O motivo da escolha dessas linguagens se dá porque
são as linguagens mais usadas para OpenCV, a melhor abordagem ainda será
analisada. Para o reconhecimento dos caracteres será utilizado o software Tesseract, havendo a possibilidade de precisar ser treinado para reconhecer a fonte da placa de transito brasileira. As escolhas a serem
feitas tem como objetivo maximizar os resultados ao final do trabalho, tentando
criar um balanço entre facilidade de implementação e qualidade do
reconhecimento.

\section{Aquisição das imagens}
\label{sec:aquisicao}

\section{Extração da placa}
\label{sec:extracao}

\section{Segmentação dos caracteres}
\label{sec:segmentacao}

O terceiro passo para o reconhecimento da placa é a segmentação dos caracteres. A segmentação dos caracteres consiste na extração dos caracteres utilizando estratégias como projetar as suas informações de cores, rotulá-los ou comparar suas posições com modelos. A placa extraida no passo anterior pode conter problemas de inclinação ou iluminação, mas o algoritmo de segmentação deve superar todos esses problemas com preprocessamento. ~\cite{s2013automatic}

~\cite{s2013automatic} faz uma análise dos algoritmos de segmentação mais utilizados com seus prós e contras. Os principais algoritmos utilizados são: segmentação utilizando conectividade de pixels, segmentação utilizando perfis de projeção, segmentação utilizando conhecimento anterior dos caracteres, segmentação utilizando contorno dos caracteres e segmentação utilizando características combinadas.

Analisando os resultados foi definido que a segmentação dos caracteres utilizando perfis de projeção foi o mais eficiente, e que mais se encaixa no problema proposto.~\cite{sanyuan2004car} utiliza essa técnica de segmentação de caracteres, e, utilizando-a juntamente de remoção de ruidos e análise de sequencia de caracateres,obteve uma taxa de acerto de 99.2\% e uma velocidade de processamento de 10 a 20 milisegundos, o que é bem animador. As vantagens deste método é que a segmentação independe das posições dos caracteres, e consegue lidar bem com rotações. Suas desvantagens são que é afetada ruído na imagem e requer o conhecimento do número de caracteres na placa, o que não será problema devido ao fato de as placas brasileiras terem um número constante de caracteres.

Este método utiliza-se da diferença entre a cor dos caracteres da placa e a cor do fundo da placa, por terem valores diferentes eles tem valores binários opostos em uma imagem binária. Portanto, o método de segmentação consiste em projetar a placa extraída verticalmente para determinar o inicio e final dos caracteres e depois projetar os caracteres extraídos horizontalmente para extraír cada caracter independente.

~\cite{s2013automatic} ainda afirma que é evidente que este metodo de segmentação é o mais comum e o mais simples presente na literatura.

\section{Reconhecimento dos caracteres}
\label{sec:reconhecimento}
