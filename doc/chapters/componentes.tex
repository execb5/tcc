Para o desenvolvimento do trabalho serão utilizados alguns componentes externos,
tanto de \emph{hardware} quanto de \emph{software}. Será utilizado o computador
\emph{Raspberry Pi} e seu módulo de camera. Será utilizada a linguagem de programação
\emph{Python}. Será utilizada a biblioteca de visão computacional \emph{OpenCV}
juntamente com outras bibliotecas auxiliares que permitam integrar o \emph{OpenCV}
com a linguagem de programação \emph{Python} e o módulo de camera do \emph{Raspberry Pi}.

\section{Raspberry Pi}
\label{sec:raspi}

\begin{figure}[H]
	\centering
	\includegraphics[width=88mm]{raspberrypi.jpg}
	\caption{O Raspberry Pi 3 model B}
	\label{fig:raspberrypi}
\end{figure}

Raspberry Pi é um computador construído em uma placa de circuito do tamanho de
um cartão de crédito desenvolvido pela Raspberry Pi
Foundation\footnote{https://www.raspberrypi.org/}. Existem diversos modelos
diferentes de Raspberry Pi no mercado, o que será utilizado no trabalho é um dos
mais recentes, o Raspberry Pi 3 model B. Será utilizado o sistema
operacional \emph{Raspbian}, que é o sistema operacional oficial suportado pela
\emph{Raspberry Pi Foundation}.

O computador ainda tem suas limitações, com um processador
quad-core ARMv8 de 1.2GHz e apenas 1GB de memória RAM\@. Pela limitação do
\emph{hardware} é possível que algumas aplicações fiquem mais lentas do que
ficariam em um computador mais potente, apesar disso, o Raspberry Pi é um
computador bem completo e capaz de exercer todas as funções de um computador
normal.

O modelo utilizado de Raspberry Pi contém módulo de WiFi embutido, sem a
nescessidade de periférico, que será utilizado no trabalho para enviar as
informações processadas pela Internet. Também será utilizado um módulo externo
de câmera para coletar as imagens em tempo real.

\section{OpenCV}
\label{sec:opencv}

OpenCV (\emph{Open Source Computer Vision Library}) é uma biblioteca \emph{open
source} de visão computacional e aprendizado de máquina. Contém mais de 2500
algoritmos otimizados nessas áreas, incluindo algoritmos clássicos e recentes. A
biblioteca é escrita nativamente em C++, e dispõe de interfaces para C, C++,
Python, Java e MATLAB, suportando os sistemas operacionais Windows, Linux,
Android e Mac OS.\footnote{http://opencv.org/}

No desenvolvimento deste trabalho será utilizada a linguagem de programação \emph{Python},
e algumas bibliotecas são utilizadas para trabalhar com \emph{OpenCV}. A biblioteca
\emph{numpy}\footnote{http://www.numpy.org/}, é uma biblioteca de computação científica em
\emph{Python}, que inclui funções de processamento numérico e vetores que são utilizados pelo \emph{OpenCV} para representar as imagens. 

Para as aplicações de aprendizado de máquina será utilizada a biblioteca \emph{ml} que é um módulo do \emph{OpenCV}. Esta biblioteca contém um conjunto de classes e métodos para classificação estatística, regressão e agrupamento de dados.\footnote{http://docs.opencv.org/2.4/modules/ml/doc/ml.html}

Também será utilizada o pacote \emph{picamera}\footnote{https://picamera.readthedocs.io/en/release-1.12/},
que possui uma interface em \emph{Python} para se comunicar com o módulo de camera do \emph{Raspberry Pi}.
É possível utilizar as funções próprias de captura de imagem da camera do \emph{OpenCV}, como por exemplo
o \texttt{cv2.VideoCapture(0)}, mas será optado por utilizar o \emph{picamera}, por ser uma interface mais
específica para a camera utilizada.
