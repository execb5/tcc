\documentclass[a0,portrait]{a0poster}

\usepackage{multicol}
\columnsep=100pt
\columnseprule=3pt

\usepackage{ragged2e}
\usepackage[utf8]{inputenc}		% Codificacao do documento (conversão automática dos acentos)
\usepackage{times}
\usepackage{graphicx}
\graphicspath{{../}}
\usepackage{booktabs}
\usepackage[font=small,labelfont=bf]{caption}
\usepackage{amsfonts, amsmath, amsthm, amssymb, marvosym}
\usepackage{wrapfig}
\usepackage{enumitem}
\usepackage{fancyvrb}
\usepackage{xspace}
\usepackage[svgnames,usenames,dvipsnames]{xcolor}

\usepackage{listings}
\usepackage{listingsutf8}

\lstset{literate=
	{ã}{{\~a}}1 {ẽ}{{\~e}}1 {ĩ}{{\~i}}1 {õ}{{\~o}}1 {ũ}{{\~u}}1
	{Ã}{{\~A}}1 {Ẽ}{{\~E}}1 {Ĩ}{{\~I}}1 {Õ}{{\~O}}1 {Ũ}{{\~U}}1	
	{á}{{\'a}}1 {é}{{\'e}}1 {í}{{\'i}}1 {ó}{{\'o}}1 {ú}{{\'u}}1
	{Á}{{\'A}}1 {É}{{\'E}}1 {Í}{{\'I}}1 {Ó}{{\'O}}1 {Ú}{{\'U}}1
	{à}{{\`a}}1 {è}{{\`e}}1 {ì}{{\`i}}1 {ò}{{\`o}}1 {ù}{{\`u}}1
	{À}{{\`A}}1 {È}{{\'E}}1 {Ì}{{\`I}}1 {Ò}{{\`O}}1 {Ù}{{\`U}}1
	{ä}{{\"a}}1 {ë}{{\"e}}1 {ï}{{\"i}}1 {ö}{{\"o}}1 {ü}{{\"u}}1
	{Ä}{{\"A}}1 {Ë}{{\"E}}1 {Ï}{{\"I}}1 {Ö}{{\"O}}1 {Ü}{{\"U}}1
	{â}{{\^a}}1 {ê}{{\^e}}1 {î}{{\^i}}1 {ô}{{\^o}}1 {û}{{\^u}}1
	{Â}{{\^A}}1 {Ê}{{\^E}}1 {Î}{{\^I}}1 {Ô}{{\^O}}1 {Û}{{\^U}}1
	{œ}{{\oe}}1 {Œ}{{\OE}}1 {æ}{{\ae}}1 {Æ}{{\AE}}1 {ß}{{\ss}}1
	{ű}{{\H{u}}}1 {Ű}{{\H{U}}}1 {ő}{{\H{o}}}1 {Ő}{{\H{O}}}1
	{ç}{{\c c}}1 {Ç}{{\c C}}1 {ø}{{\o}}1 {å}{{\r a}}1 {Å}{{\r A}}1
	{€}{{\EUR}}1 {£}{{\pounds}}1
}

\lstdefinestyle{codeStyle}{
	commentstyle=\color{black},
	basicstyle=\ttfamily\footnotesize,
	breakatwhitespace=false,         
	breaklines=true,                 
	captionpos=b,                    
	keepspaces=true,                 
	numbers=left,                    
	numbersep=5pt,                  
	showspaces=false,                
	showstringspaces=false,
	showtabs=false,                  
	tabsize=2
}
\renewcommand{\lstlistingname}{Código}
\renewcommand{\tablename}{Tabela}
\renewcommand{\figurename}{Figura}
\newcommand\itemadjust{\itemsep.5em \parskip0pt \parsep0pt}

% ---
% Formatação de código-fonte
% ---
\usepackage{listings}





\begin{document}

%---------------------------------------------------------------
%	POSTER HEADER 
%---------------------------------------------------------------
\begin{minipage}[c]{\linewidth}
	\vspace{0.1cm}
	\noindent\makebox[\textwidth][c]{
		
	\begin{minipage}[c]{0.15\linewidth}
		\begin{center}
			\includegraphics[width=9cm]{fig/pucrs-logo.pdf}
		\end{center}
	\end{minipage}
	
	\begin{minipage}[c]{0.70\linewidth}
		\centering
		\veryHuge \color{NavyBlue} 
		\textbf{Seu Título Aqui}\\
		
		\color{Black}
		\huge \textbf{Daniel Antoniazzi Amarante, Matthias Oliveira de Nunes}\\
		\Large \textbf{Orientador: Nome do Orientador}\\
		Curso de Ciência da Computação\\
		Pontifícia Universidade Católica do Rio Grande do Sul\\
		\Large \Letter ~ \texttt{\{daniel.amarante,matthias.nunes\}@acad.pucrs.br}\\
	\end{minipage}
	
	\begin{minipage}[c]{0.15\linewidth}
		\begin{center}
			\includegraphics[width=9cm]{fig/facin-logo.pdf}
		\end{center}
	\end{minipage}}
	\\[0.1cm]%
% A bit of extra whitespace between the header and poster content
\end{minipage}

\vspace{1cm}

%---------------------------------------------------------------

\begin{multicols}{2} 
%---------------------------------------------------------------
%	MOTIVAÇÃO
%---------------------------------------------------------------
\color{NavyBlue}
\section*{\huge Seção 1}
\color{Black}
\Large
\justifying

\begin{itemize}
	\item Ponto 1;
	\item Ponto 2;
	\item 
\end{itemize}

%---------------------------------------------------------------
%   AGENTES
%---------------------------------------------------------------
\color{NavyBlue}
\section*{\huge Seção 2}
\color{Black}

\begin{itemize}
	\item Idéia 1; e
	\item Idéia 2.
\end{itemize}

%---------------------------------------------------------------
%	AgentSpeak(L)
%---------------------------------------------------------------
\color{NavyBlue}
\section*{\huge Seção 3}
\color{Black}

\begin{itemize}
	\item É uma linguagem abstrata de programação orientada a agentes baseada na arquitetura BDI;
	\item Os programas desenvolvidos utilizando a linguagem, como exemplificado no Código~\ref{alg:exemplo-hello-world}, são especificados por um conjunto de crenças, planos, eventos ativadores e ações que o agente executa no ambiente;
	%\item Rao \cite{article:Rao:1996} introduz as noções básicas para especificação desses conjuntos a partir das definições apresentadas na Tabela~\ref{tab:sintaxe-agentsepak}.
	\item As definições da Tabela~\ref{tab:sintaxe-agentsepak} são utilizadas para especificar estes conjuntos.
\end{itemize}

\vspace{13mm}


\begin{itemize}
	\item Mais idéias...:
	\begin{enumerate}
		[leftmargin=2em]\itemadjust
		\item Idéia 1;
		\item Idéia 2; 
		\item Idéia 3; e
		\item Idéia 4.
	\end{enumerate}	
	\item A Figura~\ref{fig:minhafigura} ilustra isto.
\end{itemize}
\vspace{13mm}

\begin{center}
	%\includegraphics[width=0.99\linewidth]{fig/minha-figura.pdf}
	\Huge Figura Aqui (include comentado)
	\captionof{figure}{Exemplo de Figura.}
	\label{fig:minhafigura}
\end{center}	


%---------------------------------------------------------------
%	AgentSpeak(Py)
%---------------------------------------------------------------
\color{NavyBlue}
\section*{\huge Seção 3}
\color{Black}

\begin{itemize}
	\item Mais idéias...:
	\begin{enumerate}
		[leftmargin=2em]\itemadjust
		\item Idéia 1;
		\item Idéia 2; 
		\item Idéia 3; e
		\item Idéia 4.
	\end{enumerate}	
	\item A Figura~\ref{fig:minhafigura} ilustra isto.
\end{itemize}


\vspace{13mm}
\noindent\begin{minipage}{.235\textwidth}
	\begin{minipage}{\textwidth}
		\lstset{style=codeStyle}
		\begin{lstlisting}[language=Prolog, label={alg:exemplo-hello-world}, caption={Exemplo de programa em AgentSpeak(L).}]
		/* Agent helloWorld */
		/* Initial beliefs and rules */
		
		/* Initial goals */			
		!start.
		/* Plans */
		+!start : true <- aloha; ?continue(true);      !run (agentspeak).
		+!run(A) : true <- mahalo(A).
		\end{lstlisting}
	\end{minipage}\hfill
	\vspace{7mm}
	
	\begin{minipage}{\textwidth}
		\lstset{style=codeStyle}
		\begin{lstlisting}[language=Prolog, label={alg:exemplo-projeto-hello-world}, caption={Exemplo de projeto do AgentSpeak(Py).}]
		/* Project Name */
		helloWorld:
		// List of agents
		agents = [helloWorld]
		environment = HelloWorldEnv
		\end{lstlisting}
	\end{minipage}\hfill
\end{minipage}\hfill
\begin{minipage}{.235\textwidth}
	\lstset{style=codeStyle}
	\begin{lstlisting}[language=Python, label={alg:exemplo-environment}, caption={Exemplo da descrição do ambiente em Python.}]
	from environment import *
	lt_continue = parse_literal('continue(true)')
	
	class HelloWorldEnv(Environment):
		def __init__(self):
			Environment.__init__(self)
		
		def execute_action(self, agent_name, action):
			self.clear_perceptions()
			getattr(self, action.functor)(list(action.args))
		
		def aloha(self, *args):
			self.add_percept(lt_continue)
			print('Aloha HelloWorldEnv!')
		
		def mahalo(self, *args):
			print('Mahaloing with %s!' % ", ".join(map(str, *args)))
	\end{lstlisting}
\end{minipage}

%---------------------------------------------------------------
%	Tecnologias Utilizadas
%---------------------------------------------------------------
\color{NavyBlue}
\section*{\huge Seção 4}
\color{Black}





%---------------------------------------------------------------
%	REFERENCES
%---------------------------------------------------------------
%\vspace{-10mm}
%\large
%\color{NavyBlue}
%\color{Black}
%\raggedright
%\bibliographystyle{plain}
%\bibliography{poster}

\end{multicols}

%----------------------------------------------------------------------------------------
\end{document}
