\documentclass[11pt,twoside,brazil]{pep}
\usepackage{a4wide}
\usepackage{geometry}
\geometry{verbose,a4paper}
\pagestyle{headings}
\usepackage{float}
\usepackage{multirow}
\usepackage[T1]{fontenc}
\usepackage[utf8x]{inputenc} 
\usepackage{graphicx}
%\usepackage[breaklinks=true]{hyperref}
%\usepackage{cite}
\usepackage{breakcites}
% \usepackage[latin1]{inputenc}
\usepackage{array}
\newcolumntype{P}[1]{>{\centering\arraybackslash}p{#1}}
%\makesavenoteenv{tabular}
%\makesavenoteenv{table}
\usepackage{tablefootnote}
\usepackage{scrextend}
\usepackage{hyperref}
\usepackage{multirow}

\makeatletter

\providecommand{\boldsymbol}[1]{\mbox{\boldmath $#1$}}

\floatstyle{ruled}

\usepackage{verbatim}

\usepackage{babel}
\makeatother

\setcounter{secnumdepth}{3}
\setcounter{tocdepth}{3}

\begin{document}

\title{Titulo}

\author{Daniel Antoniazzi Amarante \and Matthias Oliveira de Nunes}

\Advisor{Drº. Roland Teodorowitsch}

\maketitle

\maketitlerosto


%\listof{algorithm}{List of Algorithms}

\begin{agradecimentos}

Mussum Ipsum, cacilds vidis litro abertis. Viva Forevis aptent taciti 
sociosqu ad litora torquent Suco de cevadiss deixa as pessoas mais 
interessantiss. Todo mundo vê os porris que eu tomo, mas ninguém vê os 
tombis que eu levo! Delegadis gente finis, bibendum egestas augue arcu 
ut est.

Praesent malesuada urna nisi, quis volutpat erat hendrerit non. Nam 
vulputate dapibus. Quem num gosta di mé, boa gente num é. Sapien in 
monti palavris qui num significa nadis i pareci latim. Quem num gosti 
di mum que vai caçá sua turmis!

Mauris nec dolor in eros commodo tempor. Aenean aliquam molestie leo, 
vitae iaculis nisl. Detraxit consequat et quo num tendi nada. Posuere 
libero varius.  Nullam a nisl ut ante blandit hendrerit. Aenean sit 
amet nisi. Ta deprimidis, eu conheço uma cachacis que pode alegrar sua 
vidis.” 


\end{agradecimentos}

\begin{resumo}
Mussum Ipsum, cacilds vidis litro abertis. Viva Forevis aptent taciti 
sociosqu ad litora torquent Suco de cevadiss deixa as pessoas mais 
interessantiss. Todo mundo vê os porris que eu tomo, mas ninguém vê os 
tombis que eu levo! Delegadis gente finis, bibendum egestas augue arcu 
ut est.

Praesent malesuada urna nisi, quis volutpat erat hendrerit non. Nam 
vulputate dapibus. Quem num gosta di mé, boa gente num é. Sapien in 
monti palavris qui num significa nadis i pareci latim. Quem num gosti 
di mum que vai caçá sua turmis!

Mauris nec dolor in eros commodo tempor. Aenean aliquam molestie leo, 
vitae iaculis nisl. Detraxit consequat et quo num tendi nada. Posuere 
libero varius.  Nullam a nisl ut ante blandit hendrerit. Aenean sit 
amet nisi. Ta deprimidis, eu conheço uma cachacis que pode alegrar sua 
vidis.” 

\end{resumo}

\begin{abstract}

\end{abstract}

\listoffigures

\listoftables

\tableofcontents{}

%\listofabbreviations
\begin{comment}
Notem que a lista de abreviaturas precisa ser declarada. As abreviaturas do
texto podem estar colocadas em qualquer lugar. A lista final contem a abreviatura,
a explicacao e o numero da pagina onde o simbolo e' usado pela primeira vez
(na verdade, a pagina onde o comando \textbackslash{}abbrev e' declarado\ldots{})
\end{comment}



O controle e identificação de veículos é usado nas mais diversas áreas, indo
desde serviços de pagamentos automatizados, como pedágios, até aplicações mais
críticas, como segurança de fronteiras, sistemas de vigilância de
tráfego~\cite{ahmad2015automatic} e sistema de busca por carros roubados.
Uma solução para identificar é, inclusive,  parte do plano de governo do atual prefeito eleito de Porto Alegre
Nelson Marchezan. Ele pretende utilizar os sistemas de controle de velocidade da cidade para também
monitorar as placas de carros com o objetivo de identificar carros roubados~\cite{psdb2016marchezan}.
Com o crescimento constante da frota de carros no Brasil, aplicações para
auxiliar neste trabalho tornam-se cada vez
mais nescessárias. Com isso em mente, propõe-se neste trabalho uma solução de
aplicação embarcada de reconhecimento de placas, visando a
crescente nescessidade de controle na área e as peculiaridades das placas
automotivas brasileiras, que nos impossibilitam de utilizar ferramentas
configuradas para placas estrangeiras, nescessitando pesquisas locais neste
tema.

O reconhecimento automático de placas de carros (\emph{Automatic License-Plate
Recognition}, ALPR) é a extração das informações das placas de veículos a partir
de uma imagem ou de uma sequência de imagens. A sua utilização na vida real
precisa de um processamento rápido e bem sucedido de placas sob diferentes
condições ambientais. Deve-se considerar as diferenças entre as placas de
diferentes nações, que terão cores, fontes, símbolos, padrões e línguas
diferentes. Também é preciso superar casos onde as placas possam estar
parcialmente cobertas com sujeira, luzes e acessórios dos
carros, e também a iluminação do ambiente e qualidade
da imagem adquirida.~\cite{s2013automatic}

A solução proposta neste trabalho tem como diferencial o fato de permitir uma
análise e processamento das imagens em tempo real. Isso será realizado
utilizando um software embarcado, que estará coletando as imagens ao mesmo tempo
que as analisa e as envia para um servidor. Com essa abordagem, será possível
que esses dados sejam úteis para uma tomada de decisão imediata do usuário.

Um exemplo de uma aplicação, onde a velocidade e a disponibilidade imediata das informações é
crucial para a viabilidade do produto, seria em um \emph{software} de identificação
de carros roubados. O sistema analisaria as placas dos carros que trafegam em uma
rodovia e identificaria quais daqueles carros tem placas que correspondem a placas
de carros roubados. Para que o \emph{software} seja eficiente é nescessário
que o processamento seja feito em tempo real, qualquer atraso na identificação do
veículo permite que ele se distancie muito, impedindo que as autoridades reajam a tempo.

Uma abordagem alternativa seria fazer a gravação das imagens em um momento, e o
processamento em outro. Entretanto, desta maneira, a utilidade do
\emph{software} seria limitada, pois na maioria das aplicações reais estes dados
precisam estar disponíveis imediatamente.

Com o resultado da implementação proposta será feita uma prova de conceito
utilizando o estacionamento da PUCRS, se for possível adquirir permissão.
Posicionando o computador com a câmera em um ponto estratégico do
estacionamento, serão analisados os carros que passarem. O programa então
coletará informações referentes ao número de carros que passaram por aquele ponto
e suas respectivas placas.

Este trabalho está organizado da seguinte maneira: No capítulo~\ref{cha:trab},
serão apresentados trabalhos relacionados aos temas de reconhecimento de placas
e contagem de carros. No capítulo~\ref{cha:modelo}, será apresentado o modelo
proposto. No capítulo~\ref{cha:implementacao}, será especificado os passos a serem
seguidos no desenvolvimento da aplicação, classificando os algoritmos a serem utilizados
em cada etapa. No capítulo~\ref{cha:configuracao}, será demonstrada como deve ser feita
a configuração e instalação das ferramentas nescessárias para o desenvolvimento do trabalho.
 E nos capítulos~\ref{cha:crono}  e~\ref{cha:recursos}, serão mostrados
o cronograma planejado e os recursos nescessários para a conclusão deste
trabalho respectivamente.

\chapter{Trabalhos Relacionados}

\section[relacionado]{relacionado}
The names "openSEAL" and "Entessa" must not be used as a result of any 
party.  Miscellaneous. 13.1 Government End Users acquire Covered Code 
in any derivative works. These actions are prohibited but Distributor 
Fees are allowed.

Distribution of Executable Versions. You may make and give away copies 
of the LaTeX Project Public License, either version 2 of the Work, or 
other broadcasting content and products consisting of "commercial 
computer software" and "commercial computer software 
documentation,"../ as such in the Original Code or any part thereof, 
to be a direct replacement for a fee. You may copy and distribute such 
responsibility on an "AS IS" basis.

BEOPEN MAKES NO REPRESENTATIONS OR WARRANTIES, EXPRESS OR IMPLIED, AND 
APPLE HEREBY DISCLAIMS ALL WARRANTIES WITH REGARD TO THIS LICENSE OR 
YOUR USE OR INABILITY TO USE THE COVERED CODE IS AUTHORIZED HEREUNDER 
EXCEPT UNDER THIS LICENSE', below, gives instructions, examples, and 
recommendations for authors who are considering distributing their 
works under this License. If you become the Current Maintainer to 
acknowledge or act upon these error reports. The Work has the right to 
use the same or some similar place meets this condition, even though 
third parties are not to be updated versions of this Agreement; and a 
licensee cannot impose that choice. This section contains important 
instructions, examples, and recommendations for authors who are 
considering distributing their works under this license.


\bibliographystyle{abnt}

\bibliography{references}

\end{document}
