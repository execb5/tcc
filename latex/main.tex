\documentclass[a4paper,ordem=alf]{abntex2}

%\usepackage[brazilian]{babel}
\usepackage[utf8]{inputenc}
\usepackage[T1]{fontenc}
\usepackage{float}
\usepackage{graphicx}

%\documentclass[11pt,twoside,brazil]{pep}
%\usepackage{a4wide}
%\usepackage{geometry}
%\geometry{verbose,a4paper}
%\pagestyle{headings}
%\usepackage{float}
%\usepackage{multirow}
%\usepackage[T1]{fontenc}
%\usepackage[utf8x]{inputenc} 
%\usepackage{graphicx}
%%\usepackage[breaklinks=true]{hyperref}
%%\usepackage{cite}
%\usepackage{breakcites}
%% \usepackage[latin1]{inputenc}
%\usepackage{array}
%\newcolumntype{P}[1]{>{\centering\arraybackslash}p{#1}}
%%\makesavenoteenv{tabular}
%%\makesavenoteenv{table}
%\usepackage{tablefootnote}
%\usepackage{scrextend}
%\usepackage{hyperref}
%\usepackage{multirow}

%\makeatletter

%\providecommand{\boldsymbol}[1]{\mbox{\boldmath $#1$}}

%\floatstyle{ruled}

%\usepackage{verbatim}

%\usepackage{babel}
%\makeatother

%\setcounter{secnumdepth}{3}
%\setcounter{tocdepth}{3}

\begin{document}

\title{Título}

\author{Daniel Antoniazzi Amarante \and Matthias Oliveira de Nunes}

%\professor{Drº. Roland Teodorowitsch}

\maketitle

%\maketitlerosto


%\listof{algorithm}{List of Algorithms}


\begin{abstract}
 aeuiF6hiaehfuiaehf
\end{abstract}

\listoffigures

\listoftables

\tableofcontents{}

\chapter{Introdução}

Os sistemas de auxílio aos motoristas para encontrar vagas em estacionamentos
são algo que vem crescendo no mercado. Utilizando-se de sensores posicionados em
cada vaga, já vemos na maioria dos estacionamentos de shopping esse auxílio para
os motoristas. Porém a compra e instalação de tais sistema é bem custosa, tanto
na compra do hardware quanto na instalação(citation needed), o que nos leva a
pensar que podem haver maneiras diferentes de fazer este serviço com eficiência
e menos custo. Neste trabalho propomos a detecção de vagas e informações sobre
lotamento e identificação dos carros baseado em cameras posicionadas em pontos
de gargalo das vias do estacionamento. 

\bibliographystyle{abnt}

\bibliography{references}

\end{document}
